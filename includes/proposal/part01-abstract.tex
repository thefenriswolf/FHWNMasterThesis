%!TEX root = ../../proposal.tex

\newcommand*{\AbstractHead}[1]{%
	{\noindent\color{header-blue}\Large\textbf{#1}}
	\vspace{10pt}\\
}% These command is used to create the Date and Signature fields.

\newcommand*{\SomeSpace}{%
	\vspace{\baselineskip}
}

\AbstractHead{Abstract}
\noindent
\normalsize
\begin{body}
	Segmentation is a common task for analysis and evaluation of $\mu$CT data. Open-source software like \texttt{3D Slicer} or \texttt{ITK-SNAP} provide users with a plethora of segmentation tools, the selection of the correct algorithm for a given task has a non-negligible influence on outcome and required effort by the user. Furthermore, because 3D Slicer requires approximately 10 times more random-access-memory (RAM) than the size of the loaded dataset, user experience also plays a role in picking a segmentation algorithm compatible with the available amount of RAM.
	Despite all these issues, there is a noticeable lack of instructions, guides, or novice user approachable tutorials.

\end{body}
\SomeSpace
\AbstractHead{Keywords}
\label{s:Keywords}
\normalsize
\noindent
\begin{body}
	\noindent
	$\mu$CT, Segmentation, ITK-SNAP, 3D Slicer
\end{body}
\glsresetall
\SomeSpace

\SomeSpace
\normalsize

\glsresetall

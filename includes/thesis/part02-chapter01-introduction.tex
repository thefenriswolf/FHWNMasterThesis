%!TEX root = ../../thesis.tex

\chapter{Introduction}
\label{c:introduction}
% \section{Section Title}
% \subsection{Subsection}
% \subsubsection{Subsubsection}
% \paragraph{Paragraph}
% \subparagraph{Subparagraph}

This chapter is intended to provide an introduction to the thesis.
In \cref{s:introduction-motivation} the general motivation is discussed.
The section is followed by outlining the research goals in \cref{s:introduction-researchgoals}.
\Cref{s:introduction-researchmethodology} gives an insight into research methods used in order to answer the raised research questions.
Finally, \cref{s:introduction-structureofthisthesis} gives an outlook of the structure of the remaining thesis.

\section{Motivation}
\label{s:introduction-motivation}
Micro computed tomography (MicroCT) is a commonly used modality in industry and research.
For research, it enables the study of small animal morphology\cite{percianoInsight3DMicroCT2017}.
A common task when conducting short-term or long-term studies with mice, is the segmentation of the image datasets, as it plays an important role for quantitative image analysis\cite{sheppardTechniquesHelicalScanning2014}.
Segmentation may be defined as the process of separating various image components.
Furthermore, extracting parts relevant for subsequent analysis and comparison\cite{percianoInsight3DMicroCT2017}.
A common task may be in-vivo bone analysis for the investigation and monitoring of progression of various bone pathologies.
Analysis is performed by first segmenting the MicroCT volume.
Secondly, morphological features relevant for the study can be extracted to be then finally analyzed.
Streamlining the segmentation process is therefore crucial for the analysis step\cite{percianoInsight3DMicroCT2017,korfiatisIndependentActiveContours2017}.
\newline
For performing MicroCT dataset segmentation tasks, there exist a few capable free and open-source programs.
Two actively maintained and well-known\cite{virziComprehensiveReview3D2020,mandoliniComparisonThree3D2022,virziComprehensiveReview3D2020} software packages are:
\begin{itemize}
	\item \textsc{ITK-SNAP}\cite{yushkevichUserguided3DActive2006} available under the \texttt{GNU General Public License}\cite{licenseGnuGeneralPublic1989}
	\item \textsc{3D Slicer}\cite{kikinis3DSlicerPlatform2014} available under a \texttt{BSD style license}\cite{gaudeulPublicProvisionPrivate2005}
\end{itemize}
Both programs equip the user with a sizeable number of segmentation tools, which can be grouped into manual, semi-automatic and fully automatic\cite{percianoInsight3DMicroCT2017}.
The accuracy and efficiency of the segmentation however strongly correlate with user experience and the segmentation algorithm chosen\cite{mandoliniComparisonThree3D2022,aydinRELIABILITYREPRODUCIBILITYTIMEEFFICIENCY2020,arguelloComparisonSegmentationTools2019}.
3D Slicers documentation website\cite{pinterPolymorphSegmentationRepresentation2019} explains the software and many of the available tools in the standard distribution of 3D Slicer.
But it does not feature a recommended segmentation workflow.
Despite this fact, there is a lack of general recommendations, instructions or guides on how to efficiently segment MicroCT datasets.

% Many studies have been published which try to predict the stock market movement \citep[see][]{Mittal2012a,Nguyen2015a,Pagolu2016a,Zhang2011a}.
% The \ac{EMH} states that financial market movements depend on news, current events and product releases and all these factors will have significant impact on a company's stock value
% \citep{fama1965behavior}.
% Due to the fact that news and current events are unpredictable, stock market prices follow a random walk pattern and cannot be predicted with more than \SI{50}{\percent} accuracy
% \citep{Pagolu2016a}.

% \citet{Malkiel2003} noted that with the beginning of the new millennium, financial economists believed that stock prices are at least partly predictable.
% They emphasized the behavioral and psychological elements of stock price determination.

% Nowadays many internet users are microblogging.
% Millions of messages are published daily on popular websites which provide microblogging services, such as Twitter, Tumblr and Facebook.
% These published messages describe the personal life, opinions or current issues.
% The more users post about products and services they use, the more microblogging websites become a valuable source of peoples' opinions and sentiments.
% Therefore, this data can be used for marketing, social studies and as a measure of public opinion
% \citep{Patodkar2016a, Pagolu2016a}.
% Most Twitter messages have a maximum length of 140 characters and represents the public opinion on a precise topic
% \citep{Pagolu2016a}.


\section{Research Goals}
\label{s:introduction-researchgoals}
Furthermore, segmenting large datasets, such as produced by a high resolution MicroCT scan, results in large random-access memory (RAM) demands by the segmentation software.
The 3D Slicer documentation states that the software will require approximately 10 times more memory than the loaded dataset.
The user therefore has to optimize the segmentation workflow if the software is to be used on a computer with limited amounts of RAM.
Therefore, the author hypothesizes that users of 3D Slicer will benefit from a streamlined segmentation guide, and it will improve segmentation quality as well as workflow efficiency.
As mentioned above, \texttt{3D Slicer} and \texttt{ITK-SNAP} provide the user with a sizable amount of segmentation tools. The selection of the correct segmentation algorithm and workflow have a non-negligible influence on the segmentation quality as well as the time the user has to spend on it\cite{liuSAMMSegmentAny2023}. Furthermore, segmenting large datasets, such as produced by a high resolution $\mu$CT scan results in large random-access memory (RAM) demands by the segmentation software\cite{smistadMedicalImageSegmentation2015}.
The 3D Slicer documentation states that the software will require approximately 10 times more memory than the loaded dataset \cite{slicercommunity3DSlicerImage2022,fedorov3DSlicerImage2012}.
The user therefore has to optimize the segmentation workflow if the software is to be used on a computer with limited amounts of RAM.
Therefore, the author hypothesizes that users of \texttt{3D Slicer} will benefit from a streamlined segmentation guide, and it will improve segmentation quality as well as workflow efficiency.

% According to the factors presented in \cref{s:introduction-motivation} the central research question can be formulated:
% \emph{To what extent can stock market movements be explained by the public opinion extracted from Twitter?}

% The goal of this research is to analyze the correlation between the sentiment of tweets and the share movement of automotive companies.
% This goal will be met by achieving the following objectives:

% \begin{itemize}
% 	\item \textbf{G1} - Determine companies, keywords and stock symbols to analyze
% 	\item \textbf{G2} - Gather tweets and their sentiments and stock prices
% 	\item \textbf{G3} - Comparing sentiment time series with share prices
% \end{itemize}

% Based on the definitions of goals and having the central question in mind, the following sub tasks are defined in the form of questions in order to fulfill the goals:

% \begin{itemize}
% 	\item \textbf{G1-Q1} - Which companies should be analyzed?
% 	\item \textbf{G1-Q2} - Which keywords should be used to find corresponding tweets?
% 	\item \textbf{G1-Q3} - Which company uses which stock symbol in order to retrieve share prices?
% 	\item \textbf{G2-Q4} - Why Twitter and not any other social media platform?
% 	\item \textbf{G2-Q5} - In which way can tweets be collected?
% 	\item \textbf{G2-Q6} - In which way can sentiments be determined?
% 	\item \textbf{G2-Q7} - Which sentiments are present for various companies?
% 	\item \textbf{G3-Q8} - Can the time series of sentiments explain the share prices?
% \end{itemize}

\section{Research Methodology}
\label{s:introduction-researchmethodology}
The practical work for this project will be completed by the author on his private computer, for technical information see \cref{t:computer-specs}.
\begin{table}[ht]
	\label{t:computer-specs}
	\caption{Segmentation computer specifications}
	\begin{center}
		\begin{tabular}{l l}
			\textbf{OS}        & NixOS 24.05.4798.f4c846aee8e1 (Uakari) x86\_64    \\
			\textbf{Kernel}    & Linux 6.6.50                                      \\
			\textbf{Display}   & 1920x1080 @ 60Hz                                  \\
			\textbf{CPU}       & AMD Ryzen 7 2700U with Radeon Vega Mobile Gfx ??? \\
			\textbf{GPU}       & AMD Radeon Vega 3 / 10 Graphics @ 0.40 GHz ???    \\
			\textbf{Memory}    & 64 GiB                                            \\
			\textbf{Swap}      & ? GiB                                             \\
			\textbf{Harddrive} & samsung evo??? ``sudo smartctl -a /dev/sda''
		\end{tabular}
	\end{center}
\end{table}

10 datasets from (TODO???) were acquired by the author.
TODO: insert info about datasets.
\newline
\noindent
The first step will be the segmentation of the MicroCT scans, which will serve as a ground truth and finding possible efficient workflows with the tools included in \texttt{3D Slicer}.
Then a guide will be created outlining two different possible workflows for bone segmentation.
This guide will be handed out to test candidates in conjunction with the task of using the guide to segment a specific part of a scan volume.
The candidates are asked to perform two segmentations and return their results to the author.
The first segmentation they are asked to perform prior to reading the guide.
The second segmentation they are asked to perform after reading the guide.
After receiving the results, the author will then calculate their accuracy by comparing the result of each test candidate with his own segmentation.
While also measuring the potential improvement in segementation quality between the guided segmentation by the testers versus the unguided one.
A qualitative score of the segmentation will be created by computing a score like the Dice coefficient\cite{diceMeasuresAmountEcologic1945}.
This will be done using the \texttt{SlicerRT}\cite{pinterSlicerRTRadiationTherapy2012} extension for the 3D Slicer software package.

% The research follows a structure deducted from ``evaluation techniques for systems analysis and design modelling methods'' by \citet{Siau2011} in which the authors try to show the benefits and the shortcomings of different methods.
% In the following the three main categories and their mapping to this thesis are shown:

% \begin{description}
% 	\item[The \emph{theoretical and conceptual inquiry}]
% 	      establishes the theoretical background of this thesis.
% 	      Through literature research definitions and types of stock market prediction, option mining and social networks are found.

% 	\item[The \emph{case study}]
% 	      is needed to capture tweets on the internet.
% 	      This is done by using a \ac{DMITCAT} installation
% 	      \citep{Borra2014}.

% 	\item[The \emph{metrics analysis}]
% 	      is used to compare the results of the case study with share prices of the automotive companies.
% \end{description}

% As this thesis covers sentiments of people in a global context which are then compared to share prices in an economic context it can be classified as social science \citep{Recker2013}.
% In the following the research actions which will be undertaken to answer the questions and fulfill the goals are explained.

% \begin{itemize}
% 	\item To find answers to the questions \textbf{Q1} to \textbf{Q5} literature research will be conducted.
% 	      A keyword search will be performed on the literature search engine \emph{Google Scholar}.
% 	      Furthermore, the library will be searched as well.
% 	      The retrieved literature is reviewed and, based on the references, new literature is obtained.

% 	\item With the theoretical background which has been obtained in answering the questions \textbf{Q1} to \textbf{Q6} a tweet collection system has been set up in order to answer the question \textbf{Q7}.
% 	      This is done by setting up an open source tweet capturing system (\ac{DMITCAT}) and evaluating the sentiment of the captured tweets.

% 	\item Question \textbf{Q8} is answered through both literature research, which has been collected for the questions \textbf{Q1} to \textbf{Q6}, and evaluated sentiments of the collected tweets for question \textbf{Q7}.
% \end{itemize}

\section{Structure of this Thesis}
\label{s:introduction-structureofthisthesis}

% This section is followed by the background \cref{c:background}, where the necessary theoretical background will be explained.
% In \cref{c:casestudy}, the setup of tweet collection will be explained and the execution documented.
% Afterwards, in \cref{c:analysis}, sentiments of collected tweets will be determined and converted into a time series which will then be compared to the time series of share prices.
% Finally, in \cref{c:conclusion} the results of this work will be summed up, and limitations and further points of interest will be pointed out.

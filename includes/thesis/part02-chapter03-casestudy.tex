%!TEX root = ../../thesis.tex

\chapter{Materials and Methods}
\label{c:mm}

% This chapter describes the setup of the case study.
% The \cref{fig:casestudy-workflow} depicts the overall workflow of this thesis.
% \begin{figure}[ht]
%   \centering
%   \smartdiagramset{back arrow disabled=true,
%     text width=.6\textwidth,
%     uniform color list=black!40 for 5 items}
%   \smartdiagram[flow diagram:vertical]{
%     {Determine companies, keywords and stock symbols to analyze},
%     {Gather data},
%     {Normalization of tweets},
%     {Determine sentiment of tweets},
%     {Comparing sentiment time series with share prices}
%   }
%   \caption{Workflow of this thesis}
%   \label{fig:casestudy-workflow}
% \end{figure}

\section{Similarity score}
\label{s:similarity-score}
For comparison of the segmentations done by testers with the ground truth segmentation, a well established metric was selected.
The \acrfull{dc} is a similarity score applicable to a multitude of datasets and has been used as such by numerous studies \cite{setiawanImageSegmentationMetrics2020,atasPerformanceEvaluationJaccardDice2023}.
Coincidentally, it is readily available in 3D Slicer by a 3rd party extension called SlicerRT \cite{pinterSlicerRTRadiationTherapy2012}.
%\subsection{Dice-Sørensen coefficient}
%\label{s:dice-coefficient}
The S\o{}rensen–Dice coefficient, also known as \acrfull{dc}, is described by the equation \cite{diceMeasuresAmountEcologic1945}:\\
\begin{math}
	DSC=\frac{2|X\cap Y|}{|X|+|Y|}
\end{math}\\
It can be used to describe the similarities of two segmented images or volumes.
It is two times the intersecting \glspl{vx} of segmentation x and y divided by the sum of \glspl{vx} of x and y.\\
When dealing with boolean data the formula can be transformed to the following equation:\\
\begin{math}
	DSC=\frac{2*TP}{2*TP+FP+FN}
\end{math} \\
It is two times the \acrfull{tp} intersection divided by two times the \acrlong{tp} plus the \acrfull{fp} plus the \acrfull{fn}.
In the context of volume comparison this can be interpreted as two times the number of overlapping \glspl{vx} between two segmentations divided by the total number of \glspl{vx} in both segmentations \cite{schelbComparisonProstateMRI2021}.

\section{Datasets}
\label{s:datasets}
All nine datasets used in this study, were acquired at the competence center for preclinical imaging and biomedical engineering at the university of applied sciences wiener neustadt.
The total body scans were performed on a Molecubes X-Cube \mct\space machine (Molecubes NV X-Cube, MOLECUBES NV, Gent, Belgium).
The specimens were non contrast enhanced mice however, remnants of contrast agent could be seen in the vesica urinaria of all scanned animals.
All scans were aquired with a tube voltage of 50\acrshort{kv} and a tube current exposure time product of 70\acrshort{mas}.
Pixel spacing, the distance between the centers of neighboring pixels in the x/y-plane, was consistent across all scans with a distance of 0.1mm and the slice thickness was set to 0.1mm resulting in isotropic \glspl{vx}.\\
A summary of all scan parameters can be seen in \cref{tab:scan-parameters}.\\
\begin{table}
	\begin{center}
		\begin{tabular}{l l}
			%\hline
			% 			\textbf{Element} & \textbf{Description} \\ \hline
			Manufacturer                  & Molecubes  \\
			Model                         & X-Cube     \\
			Tube voltage (\acrshort{kv})  & 50         \\
			Tube current (\acrshort{mas}) & 70         \\
			Exposure time (sec.)          & 42         \\
			Reconstruction diameter (mm)  & 70         \\
			Reconstruction algorithm      & \gls{isra} \\
			Matrix size (\glspl{px})      & 700x700    \\
			Bit depth (\gls{bit})         & 16         \\
			Slice thickness (mm)          & 0.1        \\
			Pixel spacing (mm)            & 0.1        \\
			%\hline
		\end{tabular}
		\caption{Dataset scan parameters}
		\label{tab:scan-parameters}
	\end{center}
\end{table}

\subsection{Dataset preparation}
\label{s:mm-dataset-preparation}
\begin{figure}[h!]
	\begin{center}
		\begin{tikzpicture}[node distance=3cm]
			% nodes
			\node[draw=black,
				text centered,
				rounded corners,
				diamond]
			(01) {rough cropping};
			\node[draw=black,
				text centered,
				rounded corners,
				trapezium, trapezium left angle=70, trapezium right angle=110,
				below of=01] (02) {background masking};
			\node[draw=black,
				text centered,
				rounded corners,
				diamond,
				below of=02] (03) {fine cropping};
			\node[draw,
				text centered,
				rounded corners,
				rectangle,
				below of=03] (04) {side indicator};
			% arrows
			\draw[-latex, draw, thick] (01) edge (02);
			\draw[-latex, draw, thick] (02) edge (03);
			\draw[-latex, draw, thick] (03) edge (04);
		\end{tikzpicture}
	\end{center}
	\caption{Dataset preparation}
	\label{fig:dataset-preparation}
\end{figure}
\noindent
As can be seen in \cref{fig:dataset-preparation} several steps of preparation were performed before the dataset was handed out to testers in conjunction with the guide.
All modifications were done using built in 3D Slicer tools.
Firstly, a rough cropping of the dataset was performed.
The goal was to reduce the size of the dataset to only include anatomical structures relevant for the segmentation assignment.
This first rough cropping operation already reduced the dataset geometric size to approximately one fifth of it's original dimensions.
Next, a background masking operation has been applied with the goal of reducing file size.
In order to achieve this, first a threshold operation was performed to segment background noise in the scan as well as anatomical structures composed of soft tissue.
The resulting segment then, was the base for a masking operation where the segmented volume was uniformly overwritten with the HU value of -1000.
Thirdly, the remaining volume was cropped again, making only minor adjustments to the ROI.
The goal was to get rid of small artifacts created by the first cropping operation.
Finally, to aid the testers in determining if they are currently working on the left or right side anatomy of the dataset, side indicators were added to the volume.
This was done, using 3D Slicers Engrave tool available via the ``SlicerSegmentEditorExtraEffects'' \cite{lassoSlicerSegmentEditorExtraEffects2024} plugin.
Effectively adding an artificial segment displaying left (L) and right (R) when viewed in 3D view.


\section{Composition of the guide}
\label{s:guide_comp}
TODO

\subsection{Pictograms}
\label{s:mm-pictograms}
The following black and white pictograms were drawn by the author themselves and were explained to readers of the guide in its introductory section:
\newline
\begin{figure}[h!]
	\begin{centering}
		\begin{subfigure}{0.5\textwidth}
			\includesvg[
				inkscapelatex=false,
				width = 0.6\linewidth
			]{guide/images/2d.svg}
			\caption{2D symbol}
			\label{fig:2d_icon}
		\end{subfigure}
		\begin{subfigure}{0.5\textwidth}
			\includesvg[
				inkscapelatex=false,
				width = 0.6\linewidth
			]{guide/images/3d.svg}
			\caption{3D symbol}
			\label{fig:3d_icon}
		\end{subfigure}
	\end{centering}
	\caption{\Cref{fig:2d_icon} and \cref{fig:3d_icon} indicate weather a tool may be used in 2D or 3D mode or view.}
\end{figure}

\begin{figure}[h!]
	\begin{centering}
		\begin{subfigure}{0.5\textwidth}
			\includesvg[
				inkscapelatex=false,
				width = 0.6\linewidth
			]{guide/images/hint.svg}
			\caption{Hint symbol}
			\label{fig:hint_icon}
		\end{subfigure}
		\begin{subfigure}{0.5\textwidth}
			\includesvg[
				inkscapelatex=false,
				width = 0.6\linewidth
			]{guide/images/irreversible.svg}
			\caption{Irreversible symbol}
			\label{fig:noundo_icon}
		\end{subfigure}
	\end{centering}
	\caption{\Cref{fig:hint_icon} indicates a hint or tip while \cref{fig:noundo_icon} informs the reader that an action may not be undone easily.}
\end{figure}

\begin{figure}[h!]
	\begin{centering}
		\begin{subfigure}{0.5\textwidth}
			\includesvg[
				inkscapelatex=false,
				width = 0.6\linewidth
			]{guide/images/performance.svg}
			\caption{Performance symbol}
			\label{fig:performance_icon}
		\end{subfigure}
		\begin{subfigure}{0.5\textwidth}
			\includesvg[
				inkscapelatex=false,
				width = 0.6\linewidth
			]{guide/images/plugin.svg}
			\caption{Plugin symbol}
			\label{fig:plugin_icon}
		\end{subfigure}
	\end{centering}
	\caption{\Cref{fig:performance_icon} informs the reader about a potentially resource intensive operation and \cref{fig:plugin_icon} indicates that some tool may only be available after installing a 3rd party add-on.}
\end{figure}
\par\medskip
\noindent
3D Slicer displays the loaded dataset in 2D slice views by default.
The user has the option to visualize the dataset in 3D via 3D Slicers volume rendering capabilities.
Considerably more useful however, is the option to display already segmented parts of the dataset in 3D view.
Most segmentation tools in 3D Slicer are only usable in 2D views and were therefore marked in the guide as such with \cref{fig:2d_icon}.
A smaller part of 3D Slicers segmentation tools can be used in 3D view and were therefore marked in the guide as such with \cref{fig:3d_icon}.
The 3D symbol, which when compared to the 2D symbol is displayed with a \acrfull{dof} effect, giving it a 3-dimensional appearance.
This is intended to draw the readers attention to 3D Slicers 3D rendering view.\\
In order to draw the readers attention to a specific recommendation on how to use a tool \cref{fig:hint_icon} was utilized.
The hint symbol can be described as a simple glowing lightbulb, intended to suggest to readers that it symbolizes an idea to consider.
It was always inserted into the guide with a small section of text explaining to the user how to ideally utilize the current tool.\\
3D Slicer allows for many operations to be undone, some operations however cannot be undone.
In most cases, this is because the original data has been overwritten by the users last operation.
Unfortunately, 3D Slicer does not explicitly inform the user which operations potentially overwrite source data and are therefor impossible to undo.
In the guide \cref{fig:noundo_icon} was used to inform the reader about such issues.
This irreversible symbol displays a counterclockwise circular arrow that is disrupted by a thunderbolt which extends from the top right to the bottom left of the symbol.
The counterclockwise circular arrow is intended to convey to the reader the idea of undoing something by ``turning back time'' and the thunderbold interrupts or blocks this action and thus give the reader the idea that something cannot be undone.\\
A similar problem occurs, when using segmentation tools that are computationally expensive.
These are tools which may work fine for most users on small datasets and lead the user to believe they will work to the same extent on larger datasets.
However, loading and segmenting large datasets with some tools, especially semi-automatic segmentation tools, will lead to crashes due to 3D Slicer running out of \acrfull{ram}.
Thus, segmentation tools which may be computationally expensive were marked with \cref{fig:performance_icon} paired with a short explanation in the guide.
The performance symbol displays a \acrfull{cpu} which is intended to remind the reader that they are using a computer that has limited computational capabilities.\\
Some segmentation tools used in the guide are only available to users after installing a third party plugin for 3D Slicer, wherever this is the case, readers were informed by an instance of \cref{fig:plugin_icon}.
The plugin symbol displays a puzzle piece and is meant to convey the idea that some functionality must be intentionally added to 3D Slicer.

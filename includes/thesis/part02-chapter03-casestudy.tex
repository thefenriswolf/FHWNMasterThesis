%!TEX root = ../../thesis.tex

\chapter{Case Study}
\label{c:casestudy}

% This chapter describes the setup of the case study.
% The \cref{fig:casestudy-workflow} depicts the overall workflow of this thesis.
% In the following sections each step is elucidated.
% Strong emphasis is laid on numbers and decisions.

% \begin{figure}[ht]
%   \centering

%   \smartdiagramset{back arrow disabled=true,
%     text width=.6\textwidth,
%     uniform color list=black!40 for 5 items}
%   \smartdiagram[flow diagram:vertical]{
%     {Determine companies, keywords and stock symbols to analyze},
%     {Gather data},
%     {Normalization of tweets},
%     {Determine sentiment of tweets},
%     {Comparing sentiment time series with share prices}
%   }

%   \caption{Workflow of this thesis}
%   \label{fig:casestudy-workflow}
% \end{figure}
\lipsum[1]

\section{Determine Companies, Keywords and Stock Symbols to Analyze}
\label{s:casestudy-companieskeywords}

% First, a list of automotive companies is needed.
% These companies must be traded on a stock exchange to perform the comparison with tweet sentiments.
% As a single company may own several car brands a list of all brands has been set up.
% The result of the analysis is depicted in \cref{tab:casestudy-brands}.
% Both brands which aren't customer-facing passenger car brands and brands which do not longer exist have been omitted.
% Furthermore, the brands have been grouped by their owning company.

% \begin{longtable}[c]{!l ^l}
% 	\hline
% 	\rowstyle{\bfseries}
% 	Car brand              & Owning Company                                                       \\ \hline
% 	\endfirsthead
% 	%
% 	\multicolumn{2}{c}%
% 	{{\bfseries Table \thetable{} continued from previous page}}                                  \\
% 	                       &                                                                      \\
% 	\endhead
% 	%
% 	BMW                    & BMW \citep[p.30]{BMWGroup2017}                                       \\
% 	Mini                   & BMW  \citep[p.30]{BMWGroup2017}                                      \\
% 	Rolls-Royce            & BMW \citep[p.30]{BMWGroup2017}                                       \\
% 	Mercedes-AMG           & Daimler \citep[p.90]{DaimlerAG2018}                                  \\
% 	Mercedes-Benz          & Daimler \citep[p.90]{DaimlerAG2018}                                  \\
% 	Mercedes-Maybach       & Daimler \citep[p.90]{DaimlerAG2018}                                  \\
% 	Smart                  & Daimler \citep[p.90]{DaimlerAG2018}                                  \\
% 	Alfa Romeo             & Fiat Chrysler Automobiles \citep[p.32]{FiatChryslerAutomobiles2018a} \\
% 	Chrysler               & Fiat Chrysler Automobiles \citep[p.32]{FiatChryslerAutomobiles2018a} \\
% 	Dodge                  & Fiat Chrysler Automobiles \citep[p.32]{FiatChryslerAutomobiles2018a} \\
% 	Fiat                   & Fiat Chrysler Automobiles \citep[p.32]{FiatChryslerAutomobiles2018a} \\
% 	Fiat Professional      & Fiat Chrysler Automobiles \citep[p.32]{FiatChryslerAutomobiles2018a} \\
% 	Jeep                   & Fiat Chrysler Automobiles \citep[p.32]{FiatChryslerAutomobiles2018a} \\
% 	Lancia                 & Fiat Chrysler Automobiles \citep[p.32]{FiatChryslerAutomobiles2018a} \\
% 	RAM                    & Fiat Chrysler Automobiles \citep[p.32]{FiatChryslerAutomobiles2018a} \\
% 	Ford                   & \ford{} \citep[p.18]{FordMotorCompany2018}                           \\
% 	Lincoln                & \ford{} \citep[p.18]{FordMotorCompany2018}                           \\
% 	Baojun                 & \gm{} \citep[p.1]{GeneralMotorsCompany2018}                          \\
% 	Buick                  & \gm{} \citep[p.1]{GeneralMotorsCompany2018}                          \\
% 	Cadillac               & \gm{} \citep[p.1]{GeneralMotorsCompany2018}                          \\
% 	Chevrolet              & \gm{} \citep[p.1]{GeneralMotorsCompany2018}                          \\
% 	GMC                    & \gm{} \citep[p.1]{GeneralMotorsCompany2018}                          \\
% 	Holden                 & \gm{} \citep[p.1]{GeneralMotorsCompany2018}                          \\
% 	Jiefang                & \gm{} \citep[p.1]{GeneralMotorsCompany2018}                          \\
% 	Wuling                 & \gm{} \citep[p.1]{GeneralMotorsCompany2018}                          \\
% 	Honda                  & Honda \citep[p.3]{HondaMotorCo.2017}                                 \\
% 	Hyundai                & \hyundai{} \citep[p.127]{HyundaiMotorCompany2016}                    \\
% 	KIA                    & \hyundai{} \citep[p.127]{HyundaiMotorCompany2016}                    \\
% 	Datsun                 & Nissan Motor Corporation \citep[p.5]{NissanMotorCorporation2017}     \\
% 	Infinity               & Nissan Motor Corporation \citep[p.5]{NissanMotorCorporation2017}     \\
% 	Nissan                 & Nissan Motor Corporation \citep[p.5]{NissanMotorCorporation2017}     \\
% 	Citroën                & Groupe PSA \citep[p.3]{GroupePSA2018}                                \\
% 	Opel                   & Groupe PSA \citep[p.3]{GroupePSA2018}                                \\
% 	Peugeot                & Groupe PSA \citep[p.3]{GroupePSA2018}                                \\
% 	Vauxhall               & Groupe PSA \citep[p.3]{GroupePSA2018}                                \\
% 	Alpine                 & Groupe Renault \citep[p.11]{GroupeRenault2018}                       \\
% 	Dacia                  & Groupe Renault \citep[p.11]{GroupeRenault2018}                       \\
% 	Lada                   & Groupe Renault \citep[p.11]{GroupeRenault2018}                       \\
% 	Renault                & Groupe Renault \citep[p.10]{GroupeRenault2018}                       \\
% 	Renault Samsung Motors & Groupe Renault \citep[p.10]{GroupeRenault2018}                       \\
% 	Daihatsu               & \toyota{} \citep[p.2]{ToyotaMotorCorporation2018}                    \\
% 	Lexus                  & \toyota{} \citep[p.2]{ToyotaMotorCorporation2018}                    \\
% 	Toyota                 & \toyota{} \citep[p.2]{ToyotaMotorCorporation2018}                    \\
% 	Audi                   & \vw{} \citep[p.104]{VolkswagenAktiengesellschaft2017}                \\
% 	Bentley                & \vw{} \citep[p.104]{VolkswagenAktiengesellschaft2017}                \\
% 	Bugatti                & \vw{} \citep[p.104]{VolkswagenAktiengesellschaft2017}                \\
% 	Lamborghini            & \vw{} \citep[p.104]{VolkswagenAktiengesellschaft2017}                \\
% 	Porsche                & \vw{} \citep[p.104]{VolkswagenAktiengesellschaft2017}                \\
% 	Seat                   & \vw{} \citep[p.104]{VolkswagenAktiengesellschaft2017}                \\
% 	Škoda                  & \vw{} \citep[p.104]{VolkswagenAktiengesellschaft2017}                \\
% 	Volkswagen             & \vw{} \citep[p.104]{VolkswagenAktiengesellschaft2017}                \\ \hline

% 	\caption{Automotive brands and their corresponding owning company}
% 	\label{tab:casestudy-brands}
% \end{longtable}

% According to the survey of \emph{World Motor Vehicle Production 2016} the biggest five car manufacturing companies are: Toyota, Volkswagen, Hyundai, General Motors and Ford \citep{OICA2016}.
% Therefore the study will focus on these five companies.
% The count of manufactured cars and the companies stock symbol are summarized in \cref{tab:casestudy-companies-counts-and-symbols}.
% The number of manufactured cars are from the mentioned survey \citep{OICA2016}, the stock exchange are researched by the corresponding business reports
% (\citealp{FordMotorCompany2018};
% \citealp[page 17]{GeneralMotorsCompany2018};
% \citealp[page 92]{HyundaiMotorCompany2016};
% \citealp{ToyotaMotorCorporation2018};
% \citealp[page 110]{VolkswagenAktiengesellschaft2017})
% and the containing symbols in the table are derived by literature research and using the Yahoo Finance portal (\url{https://finance.yahoo.com}).

% \begin{table}
% 	\centering
% 	\begin{tabular}[c]{!l ^r ^l ^l}
% 		\hline
% 		\rowstyle{\bfseries}
% 		Company    & \#cars     & Stock Exchange & Symbol    \\ \hline
% 		\ford{}    & 6,429,485  & New York       & F         \\
% 		\gm{}      & 7,793,066  & New York       & GM        \\
% 		\hyundai{} & 7,889,538  & Korea          & 005380.KS \\
% 		\toyota{}  & 10,213,486 & Tokyo          & 7203.T    \\
% 		\vw{}      & 10,126,281 & Frankfurt      & VOW.F     \\  \hline
% 	\end{tabular}
% 	\caption{Automotive companies and their corresponding produced cars and stock symbol}
% 	\label{tab:casestudy-companies-counts-and-symbols}
% \end{table}
\lipsum[1]

\section{Gather Data}
\label{s:casestudy-gatherdata}

% Gathering data is split into two different tasks:
% gathering tweets which is described in detail in \cref{ss:casestudy-gatherdata-tweets},
% and gather stock prices which is described in \cref{ss:casestudy-gatherdata-stockprices}.
\lipsum[1]

\subsection{Gather Tweets}
\label{ss:casestudy-gatherdata-tweets}

% A large set of tweets is needed to perform the analysis within a time frame of at least one month.
% There were several approaches to get these tweets: download tweets directly or capture tweets within the given time frame.
% As the tracking includes five companies using 23 keywords (brands) there will be a quite big amount of data.

% Several approaches have been tried to get as many tweets as possible to the given keywords, including:

% \begin{description}
% 	\item [Official Twitter Search \ac{API}]
% 	      was the first attempt.
% 	      But there were very serious limitations to the official \ac{API} that made that quite easy way impossible.
% 	      First, the standard search \ac{API} supports just a maximum count of 100 tweets;
% 	      secondly, it supports a history of only seven days;
% 	      and lastly, there were to tight rate limits defined in order gather all possible tweets of the seven days period \citep{TwitterInc.2018}.

% 	\item [Twitter search on website]
% 	      has been investigated to overcome the shortcomings of the official Twitter \ac{API} but the tweets cannot be downloaded in a easy way as the results only appear within the web browser.
% 	      A Java tool (\url{https://github.com/Jefferson-Henrique/GetOldTweets-java}) addresses this issue and exploits the Twitter search page and downloads them immediately.
% 	      But this tool did not work as expected.
% 	      Most likely Twitter made some changes to the search result page and the tool has not been adopted since \printdate{2016-04-15} \citep{Jefferson2016}.

% 	\item [\ac{DMITCAT}]
% 	      is a toolset for capturing and analyzing tweets.
% 	      It has been developed by \citeauthor{Borra2014} to support researchers around the globe and make tweet collection and analysis easier.
% 	      It makes use of the official streaming Twitter \ac{API} which follows the push principle once subscribed
% 	      \citep{Borra2014}.
% 	      \ac{DMITCAT} support various ways of data capturing:
% 	      \begin{itemize}
% 		      \item One percent random sample of all tweets passing through Twitter,
% 		      \item Streaming endpoint of the \ac{API} to track up to 400 keywords,
% 		      \item Following up to 5000 specified users, such as members of a parliament or other expert lists \citep{Borra2014}.
% 	      \end{itemize}

% \end{description}

% After exploring all of these methods \ac{DMITCAT} has been selected to capture the necessary Twitter data by using the streaming endpoint of the \ac{API} to track 23 keywords.
% The tool is not available for usage out of the box (cloud application) but must be installed as a server instance on a own machine.
% As the tool need some serious amount of capacities and need to collect tweets 24/7 it has been installed on a virtual machine in the Microsoft Azure cloud.

% The 23 keywords have been grouped by the owning company into so called query bins.
% A query bin may contain multiple keywords to track and combine all identified tweets into one single dataset \citep{Borra2014}.

% All tweets have been captured between \printdate{2018-02-28} to \printdate{2018-09-06}.
% But there were several issues using the cloud approach:

% \begin{itemize}

% 	\item By default the virtual machine in the cloud shut down every day on 7 PM.
% 	      First a problem with the tool was expected but after several days and starting the virtual machine manually the origin of the issue was discovered and solved.

% 	\item The storage space of the virtual machine initialized with 30 \ac{GB}, which was too small for the collected tweet amount.
% 	      The storage was full after approximately 14 days of data collection.
% 	      As the problem was not detected right away it took several days for identifying and fixing the issue.

% 	\item The rate limits of the \ac{API} were hit now and then in case too many tweets were published.
% 	      \ac{DMITCAT} continued to collect tweets automatically after the corresponding time window.

% 	\item New releases \ac{DMITCAT} have been published from time to time which also required a database upgrade.
% 	      As the performance was dropping the upgrade was performed in the hope of fixing the performance issue.
% 	      Furthermore, the first update was needed to enable \ac{DMITCAT} to process tweets longer than 140 characters.
% 	      The database upgrade needed to suspend the tweet collection by a serious amount of time.
% 	      To collect as many tweets as possible the upgrade was performed in steps for each query bin separately.
% 	      The larger the query bin the longer the process took as every single tweet was needed to be altered.

% \end{itemize}

% The number of collected tweets can be seen in \cref{tab:casestudy-companies-numberoftweets}.

% \begin{table}[hbt]
% 	\centering
% 	\begin{tabular}{!l ^r ^r}
% 		\hline
% 		\rowstyle{\bfseries}
% 		Company    & \# captured tweets & \# English tweets \\ \hline
% 		\ford{}    & \num{4518198}      & \num{3745447}     \\  % has been 3745490
% 		\gm{}      & \num{575547}       & \num{413817}      \\
% 		\hyundai{} & \num{1895306}      & \num{697221}      \\  % has been 697277
% 		\toyota{}  & \num{915868}       & \num{488913}      \\
% 		\vw{}      & \num{8244083}      & \num{6219350}     \\ \hline  % has been 6219786
% 		Total      & \num{16149002}     & \num{11565283}    \\ \hline
% 	\end{tabular}

% 	\caption{Numbers of collected tweets}
% 	\label{tab:casestudy-companies-numberoftweets}
% \end{table}
\lipsum[1]

\subsection{Gather Stock Prices}
\label{ss:casestudy-gatherdata-stockprices}

% The stock data can be downloaded at any point of time for the given research period on a daily frequency basis using the Yahoo Finance website.
% The stock prices in a time frame of a year have been downloaded which contains the period of tweet collection.

% The data contains the following information for each day:
% Date, Open, High, Low, Close, Adj. Close and Volume.

% \begin{description}
%   \item [Date] on which the data applies
%   \item [Open]
% \end{description}
\lipsum[1]

\section{Normalization of Tweets}
\label{s:casestudy-normalization}

% Prior analysis the tweets have been preprocessed which is also called normalization.
% The importance of normalization has been described in \cref{ss:background-optionmining-textfeatureextraction}.

% In the following the steps of normalization are described in detail:

% \begin{description}

% 	\item [Lower case:]
% 	      The tweet text is converted to lower case as the meaning stays the same but the dictionary size is reduced.

% 	\item [Tokenization:]
% 	      For tokenization the \emph{TweetTokenizer} of the \ac{NLTK} package is used.
% 	      This tokenizer has been created for this purpose.

% 	\item [Stopwords:]
% 	      The predefined English stopwords of the \ac{NLTK} package have been removed from all tweets.

% 	\item [Lemmatization:]
% 	      There are several lemmatizer available and the \emph{WordNetLemmatizer} of the \ac{NLTK} package is used to perform this step.

% 	\item [Stemming:]
% 	      The same situation applies to stemmers.
% 	      The \emph{SnowballStemmer} of the \ac{NLTK} package is used as many found examples used this specific stemmer.

% 	\item [Custom \ac{RegEx}:]
% 	      Last but not least, custom \ac{RegEx} have been used to remove or replace other unwanted elements of tweets, such as \citep{Pagolu2016a}:

% 	      \begin{enumerate}
% 		      \item User references (``@someuser'') are replaced with the generic word ``user''.
% 		      \item Specific URI are replaced with the word ``uri''.
% 		      \item Three or more consecutive characters are replaced by two consecutive characters.
% 		            For example: the word ``coooooooool'' becomes ``cool''.
% 		            %\item Questionmarks are replaced by the word ``question'' and exclamationmarks are replace by the word ``exclamation''.
% 		      \item Hashtags have been removed as the hash symbol does not provide any additional information.
% 		            For example: the hashtag ``\#cool'' is replaced by ``cool''.
% 	      \end{enumerate}
% \end{description}
\lipsum[1]

% As these described normalization steps are language sensitive only English tweets are used for this and following steps.

\section{Determine Sentiment of Tweets}
\label{s:casestudy-sentiment}

% After normalization the tweets were ready for sentiment detection.
% The sentiments were detected using four different classifiers using the same workflow depicted in \cref{fig:casestudy-sentimentworkflow}, the used symbols are described in \cref{tab:casestudy-sentimentvariables} and the same pipeline configuration depicted in \cref{fig:casestudy-sentimentpipeline}.

% \begin{table}[hbt]
% 	\centering
% 	\begin{tabular}{!l ^l}
% 		\hline
% 		\rowstyle{\bfseries}
% 		Symbol & Description      \\ \hline
% 		$T_S$  & Sample tweets    \\
% 		$T_C$  & Collected tweets \\
% 		$P$    & Pipeline         \\
% 		$P_T$  & Trained pipeline \\
% 		$S$    & Sentiment        \\ \hline
% 	\end{tabular}

% 	\caption{Variable definitions for sentiment detection}
% 	\label{tab:casestudy-sentimentvariables}
% \end{table}

% \begin{figure}[hbt]
% 	\centering

% 	\begin{tikzpicture}
% 		\node (ts) at (0,0) [circle, draw] {$T_S$};
% 		\node (p) at (2,0) [circle, draw] {$P$};
% 		\node (tc) at (1,2) [circle, draw] {$T_C$};
% 		\node (pt) at (3,2) [circle, draw] {$P_T$};
% 		\node (s) at (5,2) [circle, draw] {$S$};

% 		\draw[->] (ts) -- (p);
% 		\draw[->] (tc) -- (pt);
% 		\draw[->] (p) -- (pt);
% 		\draw[->] (pt) -- (s);

% 		\draw (-.7,-.7) rectangle (2.7,.7);
% 		\draw (0.1,1.1) rectangle (5.9,2.9);
% 	\end{tikzpicture}

% 	\caption{Workflow for sentiment detection}
% 	\label{fig:casestudy-sentimentworkflow}
% \end{figure}

% \begin{figure}[hbt]
% 	\centering
% 	\begin{tikzpicture}
% 		\node (cv)     at (0,0) [rectangle, draw] {Count Vectorizer};
% 		\node (tfidf)  at (4,0) [rectangle, draw] {TfIdf Transformer};

% 		\node (cf_tb)  at (9,1.5) [rectangle, draw] {TextBlob};
% 		\node (cf_nb)  at (9,0.5) [rectangle, draw] {Naive Bayes};
% 		\node (cf_me)  at (9,-0.5) [rectangle, draw] {Maximum Entropy};
% 		\node (cf_svm) at (9,-1.5) [rectangle, draw] {Support Vector Machine};

% 		\node (cf)     at (9,2.5) {Classifier};

% 		\draw (6.5,-2) rectangle (11.5,3);

% 		\draw[->] (cv) -- (tfidf);
% 		\draw[->] (tfidf) -- (6.5,0);
% 	\end{tikzpicture}

% 	\caption{Pipeline for sentiment detection}
% 	\label{fig:casestudy-sentimentpipeline}
% \end{figure}

% %buitinck2013api

% The classification is done by using the programming language \emph{Python} and the \emph{scikit-learn} which includes implementations of the several classifiers including \nb{}, \me{} and \svm{}.
% Beside the existent implementation of classifiers both the language and the project \emph{scikit-learn} are easy to use and personal experience is given.
% Furthermore, the project contains a useful construct called \emph{pipelines}.
% The basic philosophy in this project is to divide the public \ac{API} into four parts:

% \begin{description}
% 	\item [Data representation.]
% 	      In most machine learning tasks the data is modeled as a set of variables.
% 	      In supervised learning tasks the goal is to find a mapping from various input variables to some output variables.
% 	      A common way to represent such a dataset is a pair of matrices: one for the input values and one for the output values.
% 	      Within one matrix one column corresponds to one variable and one row corresponds to a specific sample of the problem.
% 	      To construct such matrices from textual data \emph{scikit-learn} provides \emph{vectorizer} objects
% 	      \citep{buitinck2013api}.

% 	\item [Estimators.]
% 	      Machine learning tasks are designed as \emph{estimator} objects which expose a \emph{fit} method.
% 	      It provides the separation of model creation and learning.
% 	      A model is initialized with specific parameters regarding the specific task which are also called \emph{hyper-parameters}.
% 	      The actual learning is provided by the \emph{fit} method which takes two matrices as input parameters at least for supervised learning algorithms.
% 	      After fitting the parameters to the training data the model is trained and can be used to perform predictions or transformations
% 	      \citep{buitinck2013api}.

% 	\item [Predictors.]
% 	      \emph{Predictors} are used to produce target variables by a given set of input variables.
% 	      The \emph{predict} method is also supplied by the same model object which has been trained before by using the \emph{fit} method.
% 	      Beside the \emph{predict} method predictors have to provide a \emph{score} method which quantifies the quality of the predictions.
% 	      For supervised learning models the \emph{score} method takes two matrices as input parameters: input data and expected output data.
% 	      Using the import data the model then predicts the output values and calculates the difference to the expected output values.
% 	      The higher the calculated score is the better is the quality of the prediction
% 	      \citep{buitinck2013api}.

% 	\item [Transformers.]
% 	      Usually data is modified or filtered before it is used for model training or prediction.
% 	      Therefore \emph{scikit-learn} provides objects with the \emph{transformer} interface which exposes a method called \emph{transform}.
% 	      The method takes an input parameter for input values and generates a transformed matrix of input values.
% 	      \emph{Transformers} can be used to scale values to the standard normal distribution
% 	      \citep{buitinck2013api}.

% \end{description}

% \emph{Scikit-learn} provides furthermore a mechanism to combine several estimators to a new estimator which than can be used such as a basic estimator.
% To provide a simple sequential workflow of feature extraction, transforming, fitting and prediction \emph{scikit-learn} provides pipeline objects.
% The pipeline is a useful construct to perform the same steps over and over in a standardized way for training and estimation
% \citep{buitinck2013api}.

% % Cross validation with GridSearchCV
% Furthermore, \emph{scikit-learn} provides some helpers to find the best hyper-parameters for the given problem.
% The user can define which values various hyper-parameters can attain and the helper then perform test runs for various combinations, calculate their score and keep acting as the best performing model.
% Therefore, this type of search is called \emph{model selection}.
% \emph{Scikit-learn} provides two different model selection helpers: \emph{GridSearchCV} and \emph{RandomizedSearchCV}.
% Whereas \emph{GridSearchCV} generates a \emph{grid} of the complete combinatorial combinations and perform the scoring with every single combination, \emph{RandomizedSearchCV} takes a fixed number of parameter combinations
% \citep{buitinck2013api}.

% The model selection helper of \emph{scikit-learn} can also perform a cross-validation.
% By default \emph{GridSearchCV} uses \emph{k-fold} cross-validation by optimizing the \emph{score} function of the estimator.
% The value which should be optimized can be changed to a variety of predefined key figures including the $F-Measure$
% \citep{buitinck2013api}.

% In this thesis the \emph{GridSearchCV} model selection helper is used.
% In the following sections each part of the pipeline and tried hyper-parameters used are described.

% \begin{description}
% 	\item[Count Vectorizer]
% 	      The count vectorizer is used to generate both a vocabulary and a numerical representation of the normalized tweets.

% 	      \begin{table}[!hbt]
% 		      \centering
% 		      \begin{tabular}{!l ^l}
% 			      \hline
% 			      \rowstyle{\bfseries}
% 			      Variable  & Tried values      \\ \hline
% 			      NGrams    & 1-gram to 4-grams \\
% 			      Stopwords & none              \\
% 			      Binary    & true, false       \\ \hline
% 		      \end{tabular}

% 		      \caption{Hyper-parameters of the CountVectorizer}
% 		      \label{tab:casestudy-hyperparams-countvectorizer}
% 	      \end{table}


% 	\item[\ac{TF-IDF} Transformer]
% 	      The \ac{TF-IDF} is used to transform the vocabulary from the count vectorizer to make common words less important for the analysis.
% 	      The implementation used also supports to deactivate the transformation at all.
% 	      If the untransformed vocabulary would lead to better results the GridSearchCV would figure that out and use the best set of parameters.
% 	      All hyper-parameters for the transformer can be found in \cref{tab:casestudy-hyperparams-tfidftransfomer}.

% 	      \begin{table}[!hbt]
% 		      \centering
% 		      \begin{tabular}{!l ^l}
% 			      \hline
% 			      \rowstyle{\bfseries}
% 			      Variable      & Tried values     \\ \hline
% 			      Use IDF       & true, false      \\
% 			      Smooth IDF    & true             \\
% 			      Normalization & 'l1', 'l2', none \\ \hline
% 		      \end{tabular}

% 		      \caption{Hyper-parameters of the \ac{TF-IDF} Transformer}
% 		      \label{tab:casestudy-hyperparams-tfidftransfomer}
% 	      \end{table}


% 	\item[Classifier]

% 	      The classifiers under investigation are \tb{}, \nb{}, \me{} and \svm{} whereas the \tb{} classifier provides the baseline for the evaluation.
% 	      The other three classifiers has been selected using the article from \citet{Pang2002} as they are known within the sentiment classification research.
% 	      The pipeline performance will be investigated using these four different classifiers and each of them has different hyper-parameters.
% 	      Therefore, the hyper-parameters will be discussed in the following sections.

% 	      \begin{description}
% 		      \item[TextBlob Classifier]

% 		            The \tb{} classifier is a pre-trained classifier which has no hyper-parameters at all.
% 		            It works out of the box.
% 		            The only effort made was to make the \tb{} classifier available to the scikit-learn package by implementing the estimator interface (the method \emph{predict}).

% 		            Therefore this classifier corresponds as baseline and the other classifiers will be compared to the performance of the \tb{} classifier.

% 		      \item[Naive Bayes Classifier]

% 		            The \nb{} classifier was the first classifier which supported hyper-parameters.
% 		            As the philosophy of \emph{scikit-learn} is to provide good default values only some hyper-parameters have been modified
% 		            \citep{buitinck2013api}.
% 		            The implementation used is class \emph{sklearn.naive\_bayes.MultinomialNB}.
% 		            All tried hyper-parameter values are depicted in \cref{tab:casestudy-hyperparams-nbclassifier}.

% 		            \begin{table}[!hbt]
% 			            \centering
% 			            \begin{tabular}{!l ^l}
% 				            \hline
% 				            \rowstyle{\bfseries}
% 				            Variable & Tried values         \\ \hline
% 				            Alpha    & $10^{-2}$, $10^{-3}$ \\ \hline
% 			            \end{tabular}

% 			            \caption{Hyper-parameters of the Naive Bayes Classifier}
% 			            \label{tab:casestudy-hyperparams-nbclassifier}
% 		            \end{table}

% 		      \item[Maximum Entropy Classifier]
% 		            The implementation used for the Maximum Entropy classifier is class \emph{sklearn.linear\_model.LogisticRegression} which have some built in solvers.
% 		            The tried solvers and other hyper-parameters can be found in \cref{tab:casestudy-hyperparams-meclassifier}.

% 		            \begin{table}[!hbt]
% 			            \centering
% 			            \begin{tabular}{!l ^l}
% 				            \hline
% 				            \rowstyle{\bfseries}
% 				            Variable   & Tried values                        \\ \hline
% 				            Solver     & 'liblinear', 'lbfgs', 'sag', 'saga' \\
% 				            Multiclass & 'auto'                              \\ \hline
% 			            \end{tabular}

% 			            \caption{Hyper-parameters of the Maximum Entropy Classifier}
% 			            \label{tab:casestudy-hyperparams-meclassifier}
% 		            \end{table}

% 		      \item[Support Vector Machine Classifier]
% 		            The Support Vector Machine classifier used is class \emph{sklearn.linear\_model.SGDClassifier} with loss function 'hinge' as this is the only support loss function which results in a linear \svm{}.
% 		            Other hyper-parameters can be found in \cref{tab:casestudy-hyperparams-svmclassifier}.

% 		            \begin{table}[!hbt]
% 			            \centering
% 			            \begin{tabular}{!l ^l}
% 				            \hline
% 				            \rowstyle{\bfseries}
% 				            Variable  & Tried values         \\ \hline
% 				            Tolerance & $10^{-2}$, $10^{-3}$ \\
% 				            Loss      & 'hinge'              \\ \hline
% 			            \end{tabular}

% 			            \caption{Hyper-parameters of the Support Vector Machine Classifier}
% 			            \label{tab:casestudy-hyperparams-svmclassifier}
% 		            \end{table}

% 	      \end{description}

% 	      A pipeline is built for each classifier.
% 	      The GridSearchCV model selector is used to train the specific pipeline using the hyper-parameters defined to find the best possible parameters for each component.

% 	      The sentiment analysis is then performed with the selected model and stored in files for further analysis.

% 	\item[Comparing Sentiment Time Series with Share Prices]

% 	      Both datasets, stock prices and sentiment analysis results respectively, are stored in files prior the comparison.
% 	      The stock prices are already in a time series format on a daily basis except for weekends or holidays but the problem of missing entries are tackled later.
% 	      First, the sentiment analysis result dataset must be condensed to form a time series for comparison.
% 	      Therefore the results per tweet are grouped per day and summed up.
% 	      As negative sentiments have the value \texttt{'-1'} and positive sentiments have the value \texttt{'1'} a number is received which is positive in case more positive than negative tweets have been published on that given day and vice versa.

% 	      Missing stock prices have been calculated iteratively and the gaps have been filled by using the following procedure:
% 	      Given $x$ is a stock price value and $y$ is the next present value with one or more values in between are missing.
% 	      The missing values are calculated using the formula $\frac{x+y}{2}$.
% 	      The formula is applied to the first missing value then the calculated value is treated as new $x$ and so on.
% 	      These steps are repeated as long values were missing
% 	      \citep{Pagolu2016a}.

% 	      As there were several phases were no tweets have been tracked only consecutive time series of at least five days have been taken into account for further analysis.
%   \end{description}
\lipsum[1]

%!TEX root = ../../thesis.tex

\chapter{Theoretical Background}
\label{c:background}

%todo:
% - supervised machine learning vs unsupervised
% 1gram, 2gram, ... ngram
% granger
% extend: s:background-socialnetworks

This section should provide the theoretical background and the foundation of the conducted study.
Therefore, this section is structured as follows: 
first, related work of stock market prediction will be presented in \cref{s:background-stockmarketprediction};
secondly, an introduction into option mining will be given in \cref{s:background-optionmining};
and thirdly, the market of social networks will be examined in \cref{s:background-socialnetworks}.

\section{Stock Market Prediction} 
\label{s:background-stockmarketprediction}


\begin{itemize}
    %\itemsep0em
    \item
        In Wea only historical information is incorporated in the current price \citep{schumaker2009textual}.

    \item
        In Semi-Stronguses historical and also currently available public information in the current price \citep{schumaker2009textual}.

    \item
        In Strong the Semi-Strong model is enhanced with currently available private information such as insider information in the share price \citep{schumaker2009textual}.
\end{itemize}

Therefore, the \ac{EMH} assumes that stock market prices are driven by new information such as news and will less depend on the current price or historical prices.

They stated that information available online may act as early indicators for changes.

These effects can be also applied to the stock markets: not just news influences the stock market but also the public opinion and mood.
Previously large surveys have been conducted to gather the public mood of a representative sample.
This was very time-consuming and expensive.
But in the last ten years a significant progress has been made in sentiment tracking techniques.
Therefore the sentiments can be extracted from news and blogs

\section{Option Mining} 
\label{s:background-optionmining}

\epigraph{We are drowning in information and starving for knowledge.}{\textit{John Naisbitt}}

Option mining is defined as extracting opinions from unstructured data using natural language processing techniques \citep[page 411]{Liu2007}.

According to Pang and Lee the two terms \emph{option mining} and \emph{sentiment analysis} are mere synonyms within this field of study \citep{Pang2008c}.
Therefore, this study will use the terms interchangeably.

There are several types of option mining available:

\begin{enumerate}
	\item 
	Sentiment classification: 
	This task is performed on a document level and classifies the text as positive or negative. 
	No further analysis is made what people may like or dislike 
	\citep[page 411]{Liu2007}.
	
	\item 
	Feature-based opinion mining and summarization: 
	This task dives deep into the text and analyzes the sentences on it's own.
	Furthermore, it discovers the opinions on objects which are liked or disliked.
	An object may be a product, service, topic or organization. 
	For example in an product review it detects the product features which have been described 
	\citep[page 412]{Liu2007}.
	
	\item
	Comparative sentence and relation mining:
	In this type of task product features are compared to the same or similar feature of another product.
	For example comparing two cameras: "the battery life of camera A is much shorter than that of camera B" 
	\citep[page 412]{Liu2007}. 
\end{enumerate}

This study will focus on short documents with given keywords in it.
Therefore, it is assumed that the documents describe the targeted topic (see \cref{s:background-socialnetworks} on page \pageref{s:background-socialnetworks} for the background).
As a result the study will focus on sentiment classification.

Sentiment classification has some similarities with topic-based text classification, which classifies the topic of documents into predefined topic classes, for example sports, science or politics.
In topic based classification topic words are important, while they are unimportant for sentiment classification \citep[page 412f]{Liu2007}.

In the following an introduction into sentiment detection techniques will be given, including:

\begin{itemize}
    \item Text Feature Extraction (see \cref{ss:background-optionmining-textfeatureextraction})
    \item Machine Learning Algorithms (see \cref{ss:background-optionmining-machinelearningalgorithms})
    \item Metrics (see \cref{ss:background-optionmining-metrics})
\end{itemize}

For these sections the variable definitions of \autoref{tab:background-optionmining-variables} apply.

\begin{table}
	\begin{center}
		\begin{tabular}{l l}
			\hline
			\textbf{Element} & \textbf{Description} \\ \hline
			$c$ & Specific document class \\
			$C$ & Document classes \\
			$d$ & Specific document \\
			$D$ & Documents \\
			$f_i$ & Specific feature \\
			$m$ & Count of features \\
			$n_i(d)$ & Count of features $i$ in document $d$ \\
			$P(x)$ & Probability of x \\ \hline
		\end{tabular}

        \caption{Variable definitions used in option mining}
        \label{tab:background-optionmining-variables}
	\end{center}
\end{table}

\subsection{Text Feature Extraction}
\label{ss:background-optionmining-textfeatureextraction}

To make text available to machine learning algorithms each document must be converted to a vector.
This section gives an introduction into text preprocessing and feature extraction techniques.

Most algorithms may yield better performance if the text is preprocessed in some way.
Preprocessing is important to reduce the number of features. Preprocessing may include:

\begin{description}
	\item [Lower case]
		Computer programs are working case sensitive by default.
		The words "good" and "Good" would yield to two different features within the opinion mining algorithms.
		As this is not expected because they are meaning the same thing and transport the same opinion the text is lower cased beforehand to reduce the size of the lexicon.

	\item [Stop words]
		Words which do not transport any opinion or sentiment are called stop words.
		In most cases these words are quite frequent within the texts.
		They can be removed as the expressed opinion should stay the same
		\citep{Nothman2018}.

		Examples for stop words are 'a', 'the' and 'for' for the English language
		\citep{schumaker2009textual}.

	\item [Lemmatization]
		Lemmatization transforms all verbs to the infinite form and nouns to their singular form 
		\citep{Shukri2015a}.
		The lemmatizer analyzes each word using vocabulary and morphological analysis and transform the words to their dictionary form
		\citep{Balakrishnan2014}.
 
	\item [Stemming]
		Stemming is used to transform words to their basic form.
		In order to achieve that the plural 's' from nouns and 'ing' from verbs are removed.
		Therefore, the words 'go' and 'going' are meaning the same thing but would yield to different features
		\citep{Shukri2015a}.

		A stemmer is either implemented using rules which are applied to the word till no further rule can be applied or the stemmer contains a list of suffixes and the longest identified suffix will be removed
		\citep{Balakrishnan2014}.
\end{description}

As most algorithms assume that the input vector is of fixed size.
The most common way to represent variable-length documents in feature vector format is to use the \ac{bow} representation
\citep{Murphy2012}.
A corpus with sample documents is depicted in \cref{tab:background-optionmining-sampledocuments}.

\begin{table}[htbp]
	\begin{center}
		\begin{tabular}{!l ^l}
			\hline
			\rowstyle{\bfseries}
			Document & Text \\ \hline
			$D_1$ & \texttt{Mary had a little lamb, little lamb, little lamb,} \\
			$D_2$ & \texttt{Mary had a little lamb, its fleece as white as snow} \\ \hline
		\end{tabular}

        \caption[An example of documents]{An example of documents, taken from \citep[p.81]{Murphy2012}}
		\label{tab:background-optionmining-sampledocuments}
	\end{center}
\end{table}

Using that representation the document is seen as a sequence of words.
Therefore, each document is divided into tokens or terms.
A token may be a word or something else like a number or a punctuation mark
\citep{Manning1999}.

The tokens are sequentially numbered and the tokens are replaced within the documents with their corresponding number.
That way the vocabulary is built (translation between number and token)
\citep{Murphy2012}.
The vocabulary is depicted in \cref{tab:background-optionmining-vocabulary} and the translated corpus is shown in \cref{tab:background-optionmining-translatedsampledocuments}.

\begin{table}[htbp]
	\begin{center}
		\begin{tabular}{!>{\bfseries}l *{6}{^>{\ttfamily}c}}
			\hline
			Number: & 1 & 2 & 3 & 4 & 5 & 6  \\
			Token: & mary & lamb & little & fleece & white & snow \\ \hline
		\end{tabular}

		\caption[A sample vocabulary]{
			A sample vocabulary;
			tokens in lower case and punctuation and stop words removed;
			deducted from \cref{tab:background-optionmining-sampledocuments}}
		\label{tab:background-optionmining-vocabulary}
		%\label{tab:background-optionmining-tokenization}
	\end{center}
\end{table}

\begin{table}[htbp]
	\begin{center}
		\begin{tabular}{!l ^l}
			\hline
			\rowstyle{\bfseries}
			Document & Token Numbers \\ \hline
			$D_1$ & \texttt{1 3 2 3 2 3 2} \\
			$D_2$ & \texttt{1 3 2 4 5 6} \\ \hline
		\end{tabular}

        \caption[An example of translated documents]{An example of translated documents. Deducted from \cref{tab:background-optionmining-sampledocuments} using vocabulary in \cref{tab:background-optionmining-vocabulary}}
		\label{tab:background-optionmining-translatedsampledocuments}
	\end{center}
\end{table}

It is assumed that a bag is not ordered.
Therefore the order of tokens is irrelevant to build a suitable representation of the text.
This is why the \ac{bow} approach is the de facto standard for financial research
\citep{schumaker2009textual}.
The next step is to count the number of tokens within the documents and build a histogram of token counts
\citep{Murphy2012}.
This is done for our sample corpus in \cref{tab:background-optionmining-termfrequencies}.

\begin{table}[htbp]
	\begin{center}
		\begin{tabular}{!>{\bfseries}l *{6}{^>{\ttfamily}c}}
			\hline
			Number: & 1 & 2 & 3 & 4 & 5 & 6  \\
			Token: & mary & lamb & little & fleece & white & snow \\ \hline
			$D_1$ & 1 & 3 & 3 & 0 & 0 & 0 \\
			$D_2$ & 1 & 1 & 1 & 1 & 1 & 1 \\ \hline
		\end{tabular}

        \caption[Sample term frequencies]{Term frequencies deducted from \cref{tab:background-optionmining-translatedsampledocuments}}
		\label{tab:background-optionmining-termfrequencies}
		%\label{tab:background-optionmining-tokenization}
	\end{center}
\end{table}

%uni-grams
%bi-grams
%tri-grams

In the previous examples each word became one token.
This approach was following the unigram (1-gram) model and can be generalized to \emph{n-gram} models.
Following the numbers one can use 2-gram, 3-gram or 4-gram models which are also referred as bigram, trigram or fourgram models.
Increasing the \texttt{n} in \emph{n-gram} models are increasing both the context and the number of tokens.
The examples of bigrams and trigrams depicted in \cref{tab:background-optionmining-ngram} are demonstrating the increased context in contrast to unigrams
\citep[p.193f]{Manning1999}.

\begin{table}[htbp]
	\begin{center}
		\begin{tabular}{!l ^p{3cm} ^p{3cm}}
			\hline
			\rowstyle{\bfseries}
			Text & Bigrams & Trigrams \\ \hline
			Mary had a little lamb & \texttt{Mary had, had a, a little, little lamb} & \texttt{Mary had a, had a little, a little lamb} \\ \hline
		\end{tabular}

        \caption{An example of bigrams and trigrams}
		\label{tab:background-optionmining-ngram}
	\end{center}
\end{table}

%bag of words
%term frequency (tf)
%inverse document frequency (idf) %TF-IDF: \citep{Bafna2016}

\subsection{Machine Learning Algorithms}
\label{ss:background-optionmining-machinelearningalgorithms}

\citeauthor{Pang2002} performed a study which analyzed several machine learning algorithms and their suitability with sentiment classification.
These three algorithms have shown a good performance in text categorization studies and therefore they tried to apply these algorithms to sentiment classification \citep{Pang2002}.

In the following each of them is introduced shortly.

\begin{description}
	\item[Naive Bayes.] 
    The \nb{} algorithm calculates the probabilities that a document $d$ belong to a class $c$.
	Furthermore, it implies that all classes are independent to each other, which does not hold true very often in a real world scenario.
	In sentiment classification tasks there are two classes available: positive and negative.
	Therefore the text is classified as class $c$ where $c* = arg max_c P(c | d)$.
	The general Bayes classifier can be seen in \autoref{eq:background-optionmining-machinelearningalgorithms-bayes} \citep{Pang2002}.
	
	\begin{equation}
		P_{NB}(c|d) = \frac{P(c) (\prod_{i=1}^{m} P(f_i|c)^{n_i(d)}) }{P(d)}
		\label{eq:background-optionmining-machinelearningalgorithms-bayes}
	\end{equation}
	
	Despite the fact that the Naive Bayes algorithm is very simple and the assumption of independent classes does not hold true in a real world scenario it performs surprisingly well \citep{Pang2002}.
	
	\item[Maximum Entropy Classification.]
	The \me{} Classification succeeded in a number of language processing applications.
	The exponential form of the Maximum Entropy Classifier can be seen in \autoref{eq:background-optionmining-machinelearningalgorithms-maximumentropy} \citep{Pang2002}.
	
	\begin{equation}
		P_{ME}(c|d) = \frac{1}{Z(d)} exp \left( \sum_i^m \lambda_{i,c}F_{i,c}(d,c) \right)
		\label{eq:background-optionmining-machinelearningalgorithms-maximumentropy}
	\end{equation}
	
	Whereas $Z(d)$ is a normalization function as depicted in \autoref{eq:background-optionmining-machinelearningalgorithms-maximumentropy_Zd} \citep{Nigam1999} and $F_{i,c}(d,c)$ is a feature/class function for feature $f_i$ and class $c$ which is defined in \autoref{eq:background-optionmining-machinelearningalgorithms-maximumentropy_fic}.

	\begin{equation}
		Z(d) = \sum_c exp(\sum_i \lambda_{i,c} F_{i,c}(d,c))
		\label{eq:background-optionmining-machinelearningalgorithms-maximumentropy_Zd}
	\end{equation}

	\begin{equation}
	F_{i,c}(d,c') = 
		\begin{cases}
		1, & n_i(d) > 0 \text{ and } c' = c \\
		0  & \text{otherwise}
		\end{cases}
		\label{eq:background-optionmining-machinelearningalgorithms-maximumentropy_fic}
	\end{equation}

	$\lambda_{i,c}$ are feature weight parameters. A larger value of that parameter means that feature $f_i$ is considered as strong indicator for class $c$.
	The values for $\lambda_{i,c}$ are set in a way that the entropy of the training dataset is maximized \citep{Pang2002}.
		
	\item[Support Vector Machine.]
	\svm{} are large-margin classifiers in contrast to the probabilistic Naive Bayes and Maximum Entropy.
	That means that in the two-category case (positive or negative) the training procedure looking for a hyperplane which maximizes the margin between the two groups \citep{Pang2002}.
	This scenario is depicted in \autoref{fig:background-optionmining-machinelearningalgorithms-svm} \citep[p. 275]{Cortes1995}.
		
	\begin{figure}[ht]
		\centering
		\includegraphics[width=.7\textwidth]{images/svm.png}
		\caption[An example of a \svm{} in a two-category case]{An example of a \svm{} in a two-category case, from \citep[p. 275]{Cortes1995}}
		\label{fig:background-optionmining-machinelearningalgorithms-svm}
	\end{figure}
	
\end{description}

\subsection{Metrics}
\label{ss:background-optionmining-metrics}

The performance of algorithms in \cref{ss:background-optionmining-machinelearningalgorithms} can be measured by some metrics.
In the following the most important metrics will be presented.

\begin{description}
	\item [Confusion Matrix]
		The confusion matrix shows the relation between correctly and wrongly predicted items.
		To build a confusion matrix each label is predicted by the algorithm and counted as follows:
		If the correct label is \emph{positive} and the algorithm predicted \emph{positive}, than it is a \ac{TP}.
		On the other hand if the algorithm predicted \emph{negative} it is a \ac{FN}.
		The same structure applies if the correct label is \emph{negative}: \ac{TN} if the algorithm predicted \emph{negative} and \ac{FP} if the algorithm predicted \emph{positive}
		\citep{Tripathy2015}.

		A sample confusion matrix is depicted in \cref{tab:background-optionmining-confusionmatrix}.

		Based on this confusion matrix some other metrics can be calculated.
		
		\begin{table}
			\begin{center}
				\begin{tabular}{l | l l}
					 & \multicolumn{2}{c}{Correct labels} \\
					Predicted labels & Positive & Negative \\ \hline
					Positive & \ac{TP} & \ac{FP} \\
					Negative & \ac{FN} & \ac{TN} \\ \hline					
				\end{tabular}
		
				\caption{A sample confusion matrix}
				\label{tab:background-optionmining-confusionmatrix}
			\end{center}
		\end{table}

	\item [Precision]
		Precision measures the exactness of the algorithm.
		It is the ratio of correctly predicted positive labels to the total number of items predicted as positive
		\citep{Tripathy2015}.

		\begin{equation}
			precision = \frac{\ac{TP}}{\ac{TP}+\ac{FP}}
			\label{eq:background-optionmining-metrics-precision}
		\end{equation}

	\item [Recall]
		Recall measures the completeness of the algorithm.
		It is the ratio of number of correctly predicted positive labels to the actual number of positive items present in the corpus
		\citep{Tripathy2015}.

		\begin{equation}
			recall = \frac{\ac{TP}}{\ac{TP}+\ac{FN}}
			\label{eq:background-optionmining-metrics-recall}
		\end{equation}

	\item [F-Measure]
		F-Measure is the harmonic mean of precision and recall.
		It has the best value at 1 and the worst value at 0
		\citep{Tripathy2015}.
		The formula is show in \cref{eq:background-optionmining-metrics-f_measure}.

		\begin{equation}
			F-Measure = \frac{2 * precision * recall}{precision + recall}
			\label{eq:background-optionmining-metrics-f_measure}
		\end{equation}

	\item [Accuracy]
		Accuracy is one of the most used metric for opinion mining algorithms.
		It is calculated as the ratio of number of correctly predicted items to the total number of available items in the corpus
		\citep{Tripathy2015}.

		\begin{equation}
			accuracy = \frac{\ac{TP} + \ac{TN}}{\ac{TP}+ \ac{TN} + \ac{FP} + \ac{FN}}
			\label{eq:background-optionmining-metrics-accuracy}
		\end{equation}

\end{description}

\section{Social networks/microblogging services}
\label{s:background-socialnetworks}

In the last years many social networks started and many of them disappeared again.
There are several types of social networks available today: some of them are built on videos others on pictures and some of them mostly on text.
Those which are built on text are more easy to analyze and searchable for researchers worldwide.
But there are some constants in this field: Twitter, to identify one of them.
Twitter has become a valuable source of opinions, data and information for various researches \citep{Barbosa2010}.

Messages on Twitter are called tweets and are limited in length (140 for tweets before \printdate{2017-11-07}; 280 characters since then, at least for English tweets \citep{Rosen2017}) and due to this fact users have to concentrate on a specific topic precisely.
Therefore, Twitter is the perfect source of the public opinion as users discussing anything on Twitter \citep{Pagolu2016a, Patodkar2016a}.


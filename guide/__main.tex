\documentclass[a4paper]{article}
\usepackage[utf8]{inputenc}
\usepackage[english]{babel} % automatisch deutsche Bezeichnungen (zB Literatur statt References)
\usepackage{csquotes}
\usepackage{minted}
\usepackage{hyperref}
% \usemintedstyle{borland}
% \usemintedstyle{bw} % black and white for printing
\usepackage{booktabs} % professionelle Tabellen
\usepackage{pdfpages} % Einbinden von PDFS
\usepackage{graphicx} % u.a. JPGs als Grafiken
\usepackage{appendix} % "Nummerierung" der Anhänge mit A B C ...
\usepackage[backend=biber,style=ieee]{biblatex}
\usepackage{tabto}
\usepackage{svg}
\svgpath{{./images/}}
\definecolor{header-blue}{RGB}{22,48,114}

% hier noch beliebige weitere packages einbinden


\title{MicroCT Segmentation Guide for Mice1}
\author{Stefan Rohrbacher}
\date{\today}


\addbibresource{MeineLiteratur.bib}
\addbibresource{../includes/medtec.bib}


\graphicspath{{./images/}}

%-------------------------------------------------------------------
\begin{document}

\makeatletter

\begin{titlepage}
	\thispagestyle{empty}
	% \tikz [remember picture, overlay] %
	% \node [shift={(1cm,-1cm)}] at (current page.north east) %
	% [anchor=north east] %
	% {\includegraphics[width=6cm]{images/fhwn-logo.png}};
	\vspace*{\dimexpr-1cm-\topmargin-\headsep-\headheight-\baselineskip}%
	\hspace*{\dimexpr-4cm-\evensidemargin-\parindent}%
	\makebox[\paperwidth][r]{\includegraphics[width=8cm, height=3cm]{../images/fhwn-logo.png}}

	\begin{center}
		{\noindent \linespread{1.3} \color{header-blue} \Huge \textbf{\@title} \par }
		\vspace{5pt}
		%{\noindent\Huge \textbf{MedTech} \par}
		%{\noindent\Huge \textbf{Research Proposal} \par}
		\vspace{20pt}
	\end{center}
	\hspace{-35mm}

	\vspace{15pt}
	\tabto{2cm}STUDENT NAME: \tabto{7cm}\textbf{Stefan Rohrbacher} \\
	\tabto{2cm}STUDENT ID: \tabto{7cm}\textbf{01607307} \\
	\tabto{2cm}COURSE NAME: \tabto{7cm}\textbf{MedTech}\\
	%%\tabto{7cm}\textbf{\institutename}
	%\tabto{7cm}\textbf{MedTech} \\
	\vspace{15pt}
	\tabto{2cm}SUPERVISOR: \tabto{7cm}\textbf{Dipl. Ing. Michael Rauter}
	\tabto{2cm}DATE OF SUBMISSION: \tabto{7cm}\textbf{\today}

	\vspace{15pt}

	\begin{figure}[h!] % h - here
		\centerline{ % center a single element on page, from: https://tex.stackexchange.com/a/4438
			\includegraphics[
				% scale=0.2
				width=\textwidth
			]{finalSeg.png}}
	\end{figure}

	\vfill

	Deutschlandsberg, \today

\end{titlepage}
\makeatother

%\renewcommand{\arraystretch}{1}

%\maketitle 

%-------------------------------------------------------------------
\pagenumbering{Roman}
\addtocounter{page}{-1} % Kompensiert die Nummerierung des PDF

% Verzeichnisse
%-------------------------------------------------------------------
\newpage
\tableofcontents
\newpage
\listoffigures
\newpage
\listoftables
\newpage

\pagenumbering{arabic}

%-------------------------------------------------------------------
% Eigentlicher Text übersichtlich halten durch einzelne Files, die hier nur zusammengeführt werden - beliebig zu erweitern

\section{Introduction}
Ein bisschen Text zur Einleitung.
Empfehlung: im Editor jeden Satz in einer neuen Zeile beginnen.
Dadurch bleibt's übersichtlich, man kann einzelne Sätze leichter auskommentieren oder verschieben und man sieht, wenn man zu lang wird.

\subsection{Conventions used in this Guide}
\begin{figure}[h!] % h - here
	\centering
	\includesvg[
		inkscapelatex=false,
		width = 32pt
	]{2d.svg}
	\caption{2D symbol}
	\label{fig:2d_icon}
\end{figure}
\noindent
The element in Figure \ref{fig:2d_icon} signifies that a tool can be used in 2D views.\newline

\begin{figure}[h!] % h - here
	\centering
	\includesvg[
		inkscapelatex=false,
		width = 32pt
	]{3d.svg}
	\caption{3D symbol}
	\label{fig:3d_icon}
\end{figure}
\noindent
The element in Figure \ref{fig:3d_icon} signifies that a tool can be used in 3D views.\newline

\begin{figure}[h!] % h - here
	\centering
	\includesvg[
		inkscapelatex=false,
		width = 55pt
	]{hint.svg}
	\caption{Hint symbol}
	\label{fig:hint_icon}
\end{figure}
\noindent
The element in Figure \ref{fig:hint_icon} signifies a tip or hint.\newline

\begin{figure}[h!] % h - here
	\centering
	\includesvg[
		inkscapelatex=false,
		width = 55pt
	]{irreversible.svg}
	\caption{Irreversible symbol}
	\label{fig:noundo_icon}
\end{figure}
\noindent
The element in Figure \ref{fig:noundo_icon} signifies to be cautious because a operation may not be undone.\pagebreak

\begin{figure}[h!] % h - here
	\centering
	\includesvg[
		inkscapelatex=false,
		width = 55pt
	]{performance.svg}
	\caption{Performance symbol}
	\label{fig:performance_icon}
\end{figure}
\noindent
The element in Figure \ref{fig:performance_icon} signifies that a operation may use a lot of resources.\newline

\begin{figure}[h!] % h - here
	\centering
	\includesvg[
		inkscapelatex=false,
		width = 55pt
	]{plugin.svg}
	\caption{Plugin symbol}
	\label{fig:plugin_icon}
\end{figure}
\noindent
The element in Figure \ref{fig:plugin_icon} signifies that a tool may only be available after installing a third party plugin.\newline

\subsection{Installation}
Basic installation instructions for 3D Slicer:
\begin{enumerate}
	\item Browse to: \url{http://www.overleaf.com}
	\item Download the installer of the latest stable release\footnote{Version 5.6.2 at the time of writing this document} for your Operating System
	\item Run the installer and follow its instructions
\end{enumerate}

\subsubsection{Installation on MacOS}
On MacOS, 3D Slicer can either be installed in two diferrent ways:\newline\newline
Either via the provided package on the downloads page, or via Homebrew: \url{https://formulae.brew.sh/cask/slicer}\newline
For the conventional installation refer to the 3DSlicer wiki: \url{https://slicer.readthedocs.io/en/latest/user_guide/getting_started.html#mac}

\subsubsection{Installation on Linux}
3D Slicer ships with all its dependencies on Windows and MacOS.
On Linux it is required to install some dependencies via your distributions package manager.
The 3D Slicer wiki gives information about required packages and their respective names on different distributions: \url{https://slicer.readthedocs.io/en/latest/user_guide/getting_started.html#linux}.
To make this process easier this guide provides a distrobox manifest file, which allows the user to create a Linux container containing all necessary dependencies without polluting the host system.
To run 3D Slicer via the container run:
\inputminted[
	%fontsize=\footnotesize, % set text size
	stripnl, % strip leading newlines
	numbers=left, % display line numbers on the left
	breaklines % break lines after spaces if necessary
]{sh}{./code/runSlicer.sh}



\subsubsection{System Requirements}

\subsection{Interface and Usage}

\subsubsection{Zitate sind für wissenschaftliche Arbeiten unerlässlich}
Damit einmal etwas zitiert ist \cite[S. 127ff.]{knuth1997}
Und noch etwas zitiert, diesmal aus einem Paper - d.h. ohne Seiten \cite{mcintosh1995}
Und wenn's mehrere Zitate sind geht's auch so \cite{mcintosh1995,knuth1997}

Für das Arbeiten mit Quellen, empfehle ich Zotero (oder Mendeley).
Natürlich geht's auch die Quellen direkt aus zB Scholar im \textit{bibtex} Format ins Verzeichnis (= *.bib-Datei) zu kopieren.

\cite{mcintosh1995,knuth1997, granville1992}

\subsubsection{Auf Grafiken kann selten verzichtet werden}
Grafiken



\subsubsection{Formeln sind in der angewandeten Informatik seltener anzutreffen}
Die folgende Formel ist als eigenständige Formel nummeriert:
\begin{equation}
	\frac{\partial^2 }{\partial x^2}  % Online Formeleditor (google!) wirkt Wunder!
\end{equation}


Formeln können aber auch direkt in den Text $\frac{\partial^2 }{\partial x^2}$, was allerdings schwer zu lesen ist.
Für die Formeln in \LaTeX gibt's im Web eigene \textit{Cheat Sheets}, schließlich wurde es extra für den Formelsatz entwickelt.


\subsubsection{Tabellen gehören ohne Gitternetzlininen (booktabs Stil)}
Tabellen im Booktabs Stil sind professioneller gestaltet und lenken nicht durch Gitternetzlinien ab. Das ist gut so, denn welche Daten haben es schon verdient, ihr Leben hinter Gittern zu verbringen?
Zum Glück, gibt's dafür Online Editoren, denn von der Usability her, ist das händische Setzen einer Tabelle eine Katastrophe.

% \begin{table}[h]

% \centering
% \begin{tabular}
%\toprule
% ID & \multicolumn{1}{l}{Farbe} & \multicolumn{1}{l}{Anteil} \\ \midrule
% 1  & rot                       & 255                        \\
% 2  & grün                      & 255                        \\
% 3  & blau                      & 0                          \\ \bottomrule
% \end{tabular}
% \caption{Tabellen werden fallweise auch oben beschriftet. Dann einfach die caption im Quellcode an die richtige Stelle verschieben.}
% \end{table}


\subsubsection{Programmcode darf nur auszugsweise in die Arbeit}

Eigentlich gehören komplette Codelistings in den Anhang.
Wenn aber in der Arbeit zB ein Algorithmus erläutert werden soll, dann gehört das natürlich direkt in die Arbeit.

\inputminted[
	fontsize=\footnotesize, % set text size
	stripnl, % strip leading newlines
	numbers=left, % display line numbers on the left
	breaklines % break lines after spaces if necessary
]{python}{./code/test.py}

Verbatim ist dabei die einfachste Variante für Programmcode.
Ausgefeilter geht's dann zB mit lstlistings zur Sache, wo auch Syntax Highlighting vorgenommen werden kann. \\

\inputminted[
	fontsize=\footnotesize, % set text size
	stripnl, % strip leading newlines
	numbers=left, % display line numbers on the left
	breaklines % break lines after spaces if necessary
]{ini}{./code/slicerbox.ini}


Müssen Klassen-/Methoden-/Variablennamen im Fließtext erwähnt werden, bietet sich ein Inline-Verb-Block an, mit dem kann auf eine \verb|Class1.Method1()| Bezug genommen werden, ohne dass diese als zu lesendes Wort missverstanden wird. %PHR

\subsubsection{Wie man mit Aufzählungen verfährt}

Nummerierte Aufzählungen werden nur dann eingesetzt, wenn wirklich eine Reihenfolge vorliegt.
Kann auch geschachtelt werden (siehe Beispiel).

\begin{enumerate}
	\item Erster Schritt
	\item Zweiter Schritt
	      \begin{enumerate}
		      \item erster Subschritt zu Schritt zwei
		      \item zweiter Subschritt zu Schritt zwei
	      \end{enumerate}
	\item Dritter Schritt
\end{enumerate}

Aufzählungen, die keine zwingende Reihenfolge wiedergeben werden durch \textit{Bullet Points} formatiert.
In folgenden Beispiel wird auf Schachtelung verzichtet, obwohl sie ohne weiteres möglich wäre.

\begin{itemize}
	\item grüner Tee
	\item schwarzer Tee
	\item weisser Tee
	\item aromatisierter Tee
	\item Kräuterteee
\end{itemize}

\subsubsection{Ein paar persönliche Tipps zum Arbeiten - bitte ohne Scheu ergänzen!}

Jeden Satz in einer eigenene Zeile - das hab ich zwar schon erwähnt, ist für mich aber eine wirkliche Hilfe, wenn man sich einmal daran gewöhnt hat.

Hervorhebungen im Text \textit{nie} mit fett oder unterstrichen, sonderm \textit{immer} mit kursivem Text.
Das kriegt man entweder durch \textit{textit} hin oder \textit{emph}

(Harte) Zeilenumbrüche macht man mit einem doppelten Backslash, Absatzumbrüche hingegen durch mehr als 2 Absatzschaltungen im Quelltext.
Die Anzahl der Leerzeilen wird dabei nicht berücksichtigt, was zur schöneren Strukturierung des Quelltexts beiträgt.

Kommentare helfen, nichts zu vergessen.
% nicht vergessen!

Quick \& Dirty -- die Rohversion der Arbeit sollte möglichst zügig verfasst werden.
Das hilft, den roten Faden nicht zu verlieren.
Überarbeiten, umformulieren, Grafiken usw einfügen und überhaupt verschönern kann man später.

Warnungen (gelbes Dreieck) in Overleaf betreffen meist \textit{overfull boxes} d.h. der Text ist zu lang.
Bis zur finalen Version können diese ignoriert werden.
Fallen sie dann tatsächlich ins Gewicht (nur dann!) kann man versuchen, diese durch manuelle Silbentrennung hinzukriegen.

Querverweise verwenden die \textit{labels} zB bei Grafiken oder Tabellen und werden mit \textit{ref} oder \textit{pageref} eingefügt.

Fast nicht zu entdecken: die kleinen Dreiecke neben den Zeilennummern in Oveleaf.
Damit kann man den Bereich \textit{einfalten}, was zur Übersicht beiträgt.

Wenn man in der Vorschau doppelt in den Text klickt, springt man im Editor auf die Stelle im LaTeX-Quellcode.

Wenn man länger AFK war, dann kann's sein, dass man aus Overleaf abgemeldet ist.
Bei mir hat's noch NIE funktionert, sich über Button neu zu verbinden - es war immer ein Aussteigen - Einsteigen notwendig.

Wer sich von den überwältigenden Möglichkeiten mit LaTeX beeindrucken lassen will, googelt nach \textit{Tikz}.
Da gibt's dann fast Nix, was es nicht gibt.

Grundsätzlich ist bei der Arbeit mit \LaTeX Google ein guter Freund.
Die Community ist riesig und was man nicht in der sehr guten Hilfe von Overleaf findet, hat in den allermeisten Fällen jemand anderer schon gepostet.









\section{Basic usage}
\noindent
From this point onwards, this guide makes the following assumptions:
\begin{enumerate}
	\item 3D Slicer has been successfully installed
    \item The data you wish to work on has already been acquired
    \item The computer running 3D Slicer has access to the data
\end{enumerate}

\subsection{Loading Data}
3D Slicer can load data in two different ways.
Loading data from file or folder or load data from folder and save to database. %\ref{fig:toolbar}:2)
The database option is advisable if you are dealing with a larger number of studies, as it enables quick switching between studies.
If only a single study needs to be loaded, usage of the database is not necessary.
Clicking on the \texttt{Add Data} Button (\ref{fig:toolbar}:2) or using the keyboard shortcut \texttt{Ctrl + o} brings up a file picker dialogue.
\begin{figure}[h!]
	\centerline{
		\includegraphics[
			% scale=0.2
			width=0.90\paperwidth
		]{filepicker.png}}
	\caption{3D Slicer file picker}
	\label{fig:filepicker}
  \end{figure}
  \begin{figure}[h!]
	\centerline{
		\includegraphics[
			% scale=0.2
			width=0.90\paperwidth
		]{filepicker2.png}}
	\caption{3D Slicer file picker options}
	\label{fig:filepicker2}
\end{figure}
\noindent
Depending on whether the data is contained in a single or multiple files, click either \texttt{Choose Directory to Add} or \texttt{Choose File(s) to Add}. Browse to your data, select it and confirm your selection by clicking \texttt{Choose}.
Show additional import options by setting a checkmark at \texttt{Show Options}.
Here it is possible to load the dataset as a Label map, force 3D Slicer to ignore similar files, automatically center the volume, ignore orientation information in the DICOM header, show or hide the volume and set the color table.
After confirming the import, depending on size of the dataset, some patience may be required. If the import was successful and the dataset has not been set to be hidden after import, 3D Slicer will automatically populate the view area.

\subsection{Saving Data}
  \begin{figure}[h!]
	\centerline{
		\includegraphics[
			% scale=0.2
			width=0.90\paperwidth
		]{saveMenu.png}}
	\caption{3D Slicer file save menu}
	\label{fig:save}
\end{figure}

% TODO: mouse 3 t folder
\pagebreak
\subsection{Addressing Performance issues}
If loading the data took a long time or scrolling in the 2D views is sluggish, %TODO: insert subsection about memory usage and performance
it might be worth considering reducing the size of your dataset.
Make sure to save after every operation to reduce the loss of progress in case of a crash.
\subsubsection{Cropping} \label{crop}
Locate the \texttt{Crop Volume} module either via the dropdown menu
% TODO: insert reference to image
\texttt{Converters -> Crop Volume} and switch to it.
\begin{figure}[h!]
	\centerline{
		\includegraphics[
			% scale=0.2
			width=0.90\paperwidth
		]{moduleSwitcher.png}}
	  \caption{3D Slicer module switcher}
	  \label{fig:mS}
\end{figure}
Or click the lens icon to use the text search (see: \cref{fig:mS})
Before any cropping can happen it is required to do some setup in the newly opened module panel.
\begin{figure}[h!]
	\centerline{
		\includegraphics[
			scale=0.8
			%width=0.90\paperwidth
		]{croppingModulePanel.png}}
	  \caption{3D Slicer cropping module panel}
	  \label{fig:cMP}
\end{figure}
See \cref{fig:cMP} for reference.
Under \texttt{IO -> Input Volume} make sure the dataset you loaded is selected.
Create a new ROI under \texttt{IO -> Input ROI} choose \texttt{Create new ROI as...} and give it a distinctive name. Also make sure it is not hidden by looking for an open eye next to \texttt{Display ROI}.
Next choose a name for your output volume under \texttt{IO -> Output volume -> Create new volume as...} and give it a distinctive name.
Under \texttt{Advanced -> Fill value} choose a HU value for everything outside your ROI. It should be easily distinguishable from the structure or tissue you are segmenting. For bone segmentation I recommend choosing -1000 HU\footnote{The HU value of air}. Checking the tickbox \texttt{Advanced -> Interpolated cropping} will ensure that the output volume has the same dimensions as the input volume.
%%TODO: detour about compression???
Also check the box: \texttt{Advanced -> Isotropic spacing} with a \texttt{Spacing scale} of 1x to ensure the voxels stay isotropic.
Now adjust your ROI in the 2D view areas by clicking and dragging the colored dots at the edges and corners of the ROI.
Crop out as much excess volume as possible without affecting the anatomy you wish to segment.
Check positioning and size of your ROI by scrolling through the 2D view on all three planes, after that click \texttt{Apply}.
This operation might require some patience depending on the size of the dataset.
Note that there will be no visual confirmation the cropping operation is finished.
However, you can check by switching to the \texttt{Data} module.
If 3D Slicer is unresponsive, the operation is not done yet.
\begin{figure}[h!]
	\centerline{
		\includegraphics[
			scale=0.9
			%width=0.90\paperwidth
		]{cropConfirm.png}}
	  \caption{3D Slicer data module}
	  \label{fig:cC}
	\end{figure}
	\pagebreak
	\newline
Under \texttt{Subject hierarchy -> Node -> Scene} (see: \cref{fig:cC}) you should see:
\begin{enumerate}
  \item the loaded dataset
  \item the newly created ROI
  \item the newly created volume
\end{enumerate}
Show your cropped volume by clicking on the closed eye on the same line as its name. Hide the ROI by clicking on the open eye next to its name.
3D slicer will then automatically hide the original dataset and populate the view area with the cropped volume dataset.

\subsubsection{Masking}\label{mask}
In the last step we cropped out the part of the source volume which does not contain any material/tissue of interest.
This step is dedicate to homogenizing the volume which was not cropped out in the last step.
\begin{figure}[h!]
	\centerline{
		\includegraphics[
			%scale=0.5
			width=1.3\textwidth
		]{thesholdMasking.png}}
	  \caption{Threshold background}
	  \label{fig:tM}
\end{figure}
Switch to the \texttt{Segment Editor} module (see: \cref{fig:tM}:1).
Make sure your cropped volume is selected as the \texttt{Source volume} (see: \cref{fig:tM}:2).
Click on the \texttt{+ Add} button (\cref{fig:tM}:3). You should see a new segment appear in the segments list below.
Double click to rename it to ``Background'' or leave its default name.
Make sure your ``Background''segment is select and then activate the \texttt{Threshold} tool (\cref{fig:tM}:4).
Shift the lower threshold limit until the background is covered by the segment color in the 2D views. Afterwards shift the upper threshold limit until your tissue of interest is definitely no longer covered by the segment color.
Before applying (\cref{fig:tM}:7) the threshold segmentation make sure that 3D Slicer is allowed to edit \texttt{Everywhere} and overwrite all other segments (see: \cref{fig:tM}:6).

\pagebreak
\begin{figure}[h!]
	\centerline{
		\includegraphics[
			scale=0.8
			%width=0.80\paperwidth
		]{scissorsMasking.png}}
	  \caption{Scissors background}
	  \label{fig:sM}
	\end{figure}
\noindent
Next, activate the \texttt{Scissors} tool (see: \cref{fig:sM}:1) with your ``Background'' segment selected.
This tool provides a simple way of selecting materials and tissue that are not easily distinguished by HU value, like the MicroCT couch and positioning tools.
Change its operation mode to \texttt{Fill inside} (\cref{fig:sM}:2) and its shape to \texttt{Rectangle} or \texttt{Free-form} (\cref{fig:sM}:3).
In the 2D views you can now click and drag to create a shape, upon releasing the left mouse button the tool will add the volume inside the shape to your ``Background'' segment.
\newline % sample hint
\newline
\begin{minipage}{0.4\textwidth}
  \begin{center}
	\includesvg[
		inkscapelatex=false,
		width = 0.6\textwidth
	]{hint.svg}
\end{center}
\end{minipage}%
%
\begin{minipage}{0.5\textwidth}
    The scissors tool ``punches'' through the volume, this makes it very easy to accidentally select something intentionally. Always make sure there is no ROI in the area on the previous or next slices and check your positioning on the other planes. In most cases it will be easier to work on small sections of your volume, rather than selecting a large volume in one single step.
\end{minipage}
\newline
\newline % sample hint

\pagebreak
\begin{figure}[h!]
	\centerline{
		\includegraphics[
			%scale=0.8
			width=1.2\textwidth
		]{maskBG.png}}
	  \caption{Mask background}
	  \label{fig:mBG}
	\end{figure}
\noindent
Activate the \texttt{Mask volume} tool (see: \cref{fig:mBG}:2) and make sure your ``Background'' segment is selected (\cref{fig:mBG}:1).
As for the operation mode, choose \texttt{Fill inside} (\cref{fig:mBG}:3).
The fill value (\cref{fig:mBG}:4) should be the same as in the cropping step (\cref{crop}).
Select your cropped source volume as the \texttt{Input Volume} (\cref{fig:mBG}:5).
For the result, create a new volume by clicking on \texttt{Output Volume -> Create new Volume as...} (\cref{fig:mBG}:6) and give it a distinctive name.
\newline % sample hint
\newline
\begin{minipage}{0.4\textwidth}
  \begin{center}
	\includesvg[
		inkscapelatex=false,
		width = 0.6\textwidth
	]{irreversible.svg}
\end{center}
\end{minipage}%
%
\begin{minipage}{0.5\textwidth}
  Always create a new volume when working with the \texttt{Mask volume} tool.
  It overrides the individual voxel intensity values, this \emph{can not be undone} as 3D Slicer does not store the original intensity values in its undo history.
\end{minipage}
\newline
\newline % sample hint
\pagebreak

\subsubsection{Cleanup}
Switch to the \texttt{Data} module.
\begin{figure}[h!]
	\centerline{
		\includegraphics[
			%scale=0.8
			width=1.2\textwidth
		]{cleanup.png}}
	  \caption{Data cleanup}
	  \label{fig:clr}
	\end{figure}
\noindent
Here we see all the items we have created so far.
\Cref{fig:clr}:1 is the dicom dataset from the MicroCT scanner.
\Cref{fig:clr}:2 is the cropping ROI created in \cref{crop} and \cref{fig:clr}:3 is the volume that resulted from the cropping operation.
The segmentation item (\cref{fig:clr}:4) holds all segmentations created in the \texttt{Segment Editor} module. So far it only holds the ``Background'' segment created in \Cref{mask}. And the final item (\cref{fig:clr}:5) is the most recent volume resulting from the \texttt{Mask volume} operation.
Items \cref{fig:clr}:1-4 can be deleted to save Harddrive space and RAM while working in 3D Slicer.
\noindent
In this example the data volume on disk could be decreased from 309 megabytes to (souce dicom data) to 15 megabytes.
Which equates to approximately a 95\% reduction in file size.
As a result of this, 3D Slicers RAM consumption has dropped from 1.3 gigabytes to 585 megabytes, which equates to approximately 55\% reduction.
See \Cref{measuements:memComp} for more information.

\pagebreak


\section{Methode}
\begin{figure}[h!] % h - here
	\centerline{
		\includegraphics[
			% scale=0.2
			width=0.95\paperwidth
		]{../images/dragon.jpg}}
	\caption{Here be dragons}
	\label{fig:slicerWelcome}
\end{figure}

\section{Quick Start Guide}
\label{qsg}

% englisch - Discussion

\section{Diskussion}

Im folgenden Abschnitt werden die Ergebnisse interpretiert und Möglichkeiten zur weiteren Forschung angeführt

\subsection{Interpretation}
Die überraschenden Ergebnisse sind folgendermaßen zu interpretieren \ldots

\subsubsection{Ausblick}
Es war nicht Teil der Arbeit xy zu untersuchen, aufgrund der bisherigen Ergebnisse scheint es jedoch lohnenswert, auch diesen Weg weiter zu verfolgen \ldots


%-------------------------------------------------------------------
% Verzeichnisse
% Literaturverzeichnis am besten mit Biblatex
\printbibliography
\newpage

%-------------------------------------------------------------------
% Anhänge - können natürlich auch wieder \includes sein. 
\begin{appendix}
	\section{Anhang: Mein toller Code}
	\ldots
	\newpage
	\section{Anhang: Meine tollen Transkripte}
\end{appendix}

\end{document}

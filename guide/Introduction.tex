\section{Preface}
Ein bisschen Text zur Einleitung.
Empfehlung: im Editor jeden Satz in einer neuen Zeile beginnen.
Dadurch bleibt's übersichtlich, man kann einzelne Sätze leichter auskommentieren oder verschieben und man sieht, wenn man zu lang wird.

\subsection{Conventions used in this Guide}
\begin{figure}[h!] % h - here
	\centering
	\includesvg[
		inkscapelatex=false,
		width = 32pt
	]{2d.svg}
	\caption{2D symbol}
	\label{fig:2d_icon}
\end{figure}
The element in Figure \ref{fig:2d_icon} signifies that a tool can be used in 2D views.\newline

\begin{figure}[h!] % h - here
	\centering
	\includesvg[
		inkscapelatex=false,
		width = 32pt
	]{3d.svg}
	\caption{3D symbol}
	\label{fig:3d_icon}
\end{figure}
The element in Figure \ref{fig:3d_icon} signifies that a tool can be used in 3D views.\newline

\begin{figure}[h!] % h - here
	\centering
	\includesvg[
		inkscapelatex=false,
		width = 55pt
	]{hint.svg}
	\caption{Hint symbol}
	\label{fig:hint_icon}
\end{figure}
The element in Figure \ref{fig:hint_icon} signifies a tip or hint.\newline

\begin{figure}[h!] % h - here
	\centering
	\includesvg[
		inkscapelatex=false,
		width = 55pt
	]{irreversible.svg}
	\caption{Irreversible symbol}
	\label{fig:noundo_icon}
\end{figure}
The element in Figure \ref{fig:noundo_icon} signifies to be cautious because a operation may not be undone.\pagebreak

\begin{figure}[h!] % h - here
	\centering
	\includesvg[
		inkscapelatex=false,
		width = 55pt
	]{performance.svg}
	\caption{Performance symbol}
	\label{fig:performance_icon}
\end{figure}
The element in Figure \ref{fig:performance_icon} signifies that a operation may use a lot of resources.\newline

\begin{figure}[h!] % h - here
	\centering
	\includesvg[
		inkscapelatex=false,
		width = 55pt
	]{plugin.svg}
	\caption{Plugin symbol}
	\label{fig:plugin_icon}
\end{figure}
The element in Figure \ref{fig:plugin_icon} signifies that a tool may only be available after installing a third party plugin.\newline


\subsubsection{Zitate sind für wissenschaftliche Arbeiten unerlässlich}
Damit einmal etwas zitiert ist \cite[S. 127ff.]{knuth1997}
Und noch etwas zitiert, diesmal aus einem Paper - d.h. ohne Seiten \cite{mcintosh1995}
Und wenn's mehrere Zitate sind geht's auch so \cite{mcintosh1995,knuth1997}

Für das Arbeiten mit Quellen, empfehle ich Zotero (oder Mendeley).
Natürlich geht's auch die Quellen direkt aus zB Scholar im \textit{bibtex} Format ins Verzeichnis (= *.bib-Datei) zu kopieren.

\cite{mcintosh1995,knuth1997, granville1992}

\subsubsection{Auf Grafiken kann selten verzichtet werden}
Grafiken



\subsubsection{Formeln sind in der angewandeten Informatik seltener anzutreffen}
Die folgende Formel ist als eigenständige Formel nummeriert:
\begin{equation}
	\frac{\partial^2 }{\partial x^2}  % Online Formeleditor (google!) wirkt Wunder!
\end{equation}


Formeln können aber auch direkt in den Text $\frac{\partial^2 }{\partial x^2}$, was allerdings schwer zu lesen ist.
Für die Formeln in \LaTeX gibt's im Web eigene \textit{Cheat Sheets}, schließlich wurde es extra für den Formelsatz entwickelt.


\subsubsection{Tabellen gehören ohne Gitternetzlininen (booktabs Stil)}
Tabellen im Booktabs Stil sind professioneller gestaltet und lenken nicht durch Gitternetzlinien ab. Das ist gut so, denn welche Daten haben es schon verdient, ihr Leben hinter Gittern zu verbringen?
Zum Glück, gibt's dafür Online Editoren, denn von der Usability her, ist das händische Setzen einer Tabelle eine Katastrophe.

% \begin{table}[h]

% \centering
% \begin{tabular}
%\toprule
% ID & \multicolumn{1}{l}{Farbe} & \multicolumn{1}{l}{Anteil} \\ \midrule
% 1  & rot                       & 255                        \\
% 2  & grün                      & 255                        \\
% 3  & blau                      & 0                          \\ \bottomrule
% \end{tabular}
% \caption{Tabellen werden fallweise auch oben beschriftet. Dann einfach die caption im Quellcode an die richtige Stelle verschieben.}
% \end{table}


\subsubsection{Programmcode darf nur auszugsweise in die Arbeit}

Eigentlich gehören komplette Codelistings in den Anhang.
Wenn aber in der Arbeit zB ein Algorithmus erläutert werden soll, dann gehört das natürlich direkt in die Arbeit.

\inputminted[
	fontsize=\footnotesize, % set text size
	stripnl, % strip leading newlines
	numbers=left, % display line numbers on the left
	breaklines % break lines after spaces if necessary
]{python}{./code/test.py}

Verbatim ist dabei die einfachste Variante für Programmcode.
Ausgefeilter geht's dann zB mit lstlistings zur Sache, wo auch Syntax Highlighting vorgenommen werden kann. \\

\inputminted[
	fontsize=\footnotesize, % set text size
	stripnl, % strip leading newlines
	numbers=left, % display line numbers on the left
	breaklines % break lines after spaces if necessary
]{ini}{./code/slicerbox.ini}


Müssen Klassen-/Methoden-/Variablennamen im Fließtext erwähnt werden, bietet sich ein Inline-Verb-Block an, mit dem kann auf eine \verb|Class1.Method1()| Bezug genommen werden, ohne dass diese als zu lesendes Wort missverstanden wird. %PHR

\subsubsection{Wie man mit Aufzählungen verfährt}

Nummerierte Aufzählungen werden nur dann eingesetzt, wenn wirklich eine Reihenfolge vorliegt.
Kann auch geschachtelt werden (siehe Beispiel).

\begin{enumerate}
	\item Erster Schritt
	\item Zweiter Schritt
	      \begin{enumerate}
		      \item erster Subschritt zu Schritt zwei
		      \item zweiter Subschritt zu Schritt zwei
	      \end{enumerate}
	\item Dritter Schritt
\end{enumerate}

Aufzählungen, die keine zwingende Reihenfolge wiedergeben werden durch \textit{Bullet Points} formatiert.
In folgenden Beispiel wird auf Schachtelung verzichtet, obwohl sie ohne weiteres möglich wäre.

\begin{itemize}
	\item grüner Tee
	\item schwarzer Tee
	\item weisser Tee
	\item aromatisierter Tee
	\item Kräuterteee
\end{itemize}

\subsubsection{Ein paar persönliche Tipps zum Arbeiten - bitte ohne Scheu ergänzen!}

Jeden Satz in einer eigenene Zeile - das hab ich zwar schon erwähnt, ist für mich aber eine wirkliche Hilfe, wenn man sich einmal daran gewöhnt hat.

Hervorhebungen im Text \textit{nie} mit fett oder unterstrichen, sonderm \textit{immer} mit kursivem Text.
Das kriegt man entweder durch \textit{textit} hin oder \textit{emph}

(Harte) Zeilenumbrüche macht man mit einem doppelten Backslash, Absatzumbrüche hingegen durch mehr als 2 Absatzschaltungen im Quelltext.
Die Anzahl der Leerzeilen wird dabei nicht berücksichtigt, was zur schöneren Strukturierung des Quelltexts beiträgt.

Kommentare helfen, nichts zu vergessen.
% nicht vergessen!

Quick \& Dirty -- die Rohversion der Arbeit sollte möglichst zügig verfasst werden.
Das hilft, den roten Faden nicht zu verlieren.
Überarbeiten, umformulieren, Grafiken usw einfügen und überhaupt verschönern kann man später.

Warnungen (gelbes Dreieck) in Overleaf betreffen meist \textit{overfull boxes} d.h. der Text ist zu lang.
Bis zur finalen Version können diese ignoriert werden.
Fallen sie dann tatsächlich ins Gewicht (nur dann!) kann man versuchen, diese durch manuelle Silbentrennung hinzukriegen.

Querverweise verwenden die \textit{labels} zB bei Grafiken oder Tabellen und werden mit \textit{ref} oder \textit{pageref} eingefügt.

Fast nicht zu entdecken: die kleinen Dreiecke neben den Zeilennummern in Oveleaf.
Damit kann man den Bereich \textit{einfalten}, was zur Übersicht beiträgt.

Wenn man in der Vorschau doppelt in den Text klickt, springt man im Editor auf die Stelle im LaTeX-Quellcode.

Wenn man länger AFK war, dann kann's sein, dass man aus Overleaf abgemeldet ist.
Bei mir hat's noch NIE funktionert, sich über Button neu zu verbinden - es war immer ein Aussteigen - Einsteigen notwendig.

Wer sich von den überwältigenden Möglichkeiten mit LaTeX beeindrucken lassen will, googelt nach \textit{Tikz}.
Da gibt's dann fast Nix, was es nicht gibt.

Grundsätzlich ist bei der Arbeit mit \LaTeX Google ein guter Freund.
Die Community ist riesig und was man nicht in der sehr guten Hilfe von Overleaf findet, hat in den allermeisten Fällen jemand anderer schon gepostet.








